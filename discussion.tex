% !TEX root = thesis.tex
\startchapter{Discussion}
\label{chap:disc}
In this chapter we will walk through the approach and discuss how the research we presented in the last three chapters (Section~\ref{ch:dis:app}).
Following, we go over some general threats to validity.
Then, we shortly describe how we see the implications of making such a recommender system available in Section~\ref{} and close with concluding remarks and some future work (Section~\ref{}).

\section{Revisiting the Approach}
\labal{ch:dis:app}


\section{Approach to Leveraging Socio-Technical Congruence}
The main contribution of this thesis lies with the implicitly described approach to creating actionable knowledge from the socio-technical congruence concept as Cataldo et al~\cite{cataldo:cscw:2006} described it in their seminal paper.
Since the research questions we formulated for this thesis only implicitly describe the approach, it is necessary to describe the approach to wrest actionable insights from the concept of socio-technical congruence.

Our approach took five steps:
\begin{enumerate}
\item Define scope of interest.
\item Define outcome metric.
\item Build social networks according to the scope in real time.
\item Build technical networks according to the scope in real time.
\item Generate actionable insights.
\end{enumerate}

\paragraph{Scope} 
Each network, social, technical and thus socio-technical, needs to build with a specific scope in mind.
For example, throughout this thesis we used software builds a focus.
This focus defines the source and communication artifacts that are used to construct the socio-technical networks. 

\paragraph{Outcome}
In order to derive useful insights from the constructed networks the scope used to construct them needs to be complimented with an outcome metric.
For example, throughout this thesis we used build success, a binary variable that states whether a build is of acceptable quality.
With an outcome measure at hand we can contrast networks to gain insights.

\paragraph{Social Networks}
After the defining the scope and outcome, the next step is to construct the socio-technical networks.
In our approach we first focus on the construction of the social part of the networks.
This involves identifying all communication channels that are programatically accessible in real time.
Some examples of communication channels that can be tapped into in real time are emails, forum style discussions, or text chats.
Gathering all to the selected scope relevant communication artifacts than yields the social-network.

\paragraph{Technical Networks}
To complement the social networks and thus create socio-technical networks we need to produce technical networks.
The main issues is not to rely on information that is only available after the work has been completed and been submitted to a version repository, but to gather information to construct networks in real time.
For instance, Blincoe et al~\cite{blincoe:cscw:2012} proposed to use Mylyn\footnote{\url{http://tasktop.com/mylyn/}} events to accomplish the extraction of interactions with source code in real time.

\paragraph{Insights}
In Chapter~\ref{chap:stc-net} and~\ref{chap:actionable} we showed how to generate recommendations by contrasting the networks according to their outcome metrics based on a binary outcome metric indicating whether a build failed.
By breaking down networks, as demonstrated in Chapter~\ref{chap:stc-net} and~\ref{chap:actionable}, and correlating the different elements with the outcome metric generates insights that can be used to improve the collaboration represented by the network.

In the next section we add to the example given in this section that are based on the chapters in this thesis arguments to the feasibility and validity of the individual steps to ground the approach in the research we conducted in this thesis.

\section{Grounding the Approach}
In this section we will discuss on how the approach described in the previous Section is grounded in the research presented in this thesis.
We will discuss and ground the feasibility of each step of the approach in the research presented in this thesis and conclude this section with a more general remark on the evidence showing the value in the approach as a whole.

\paragraph{Scope}
In Chapters~\ref{chap:soc-net}-\ref{chap:stc-net} we presented work that uses a specific focus namely the build.
Selecting a software build as the focus of building and generating insights into the socio-technical networks constructed for a given build.
With a software build representing a high level software artifact identifying this artifact is straight forward as the studies presented in this thesis sufficiently demonstrated.
Generally any scope that is easily captured within a software artifact such as a software change or a work item can serve as a scope used to construct and generate insight from socio-technical networks.

\paragraph{Outcome}
With our focus on software quality in the form of build success, we chose as outcome metric whether in the opinion of the developer a build was counted as success or not.
The studies we performed and reported on in Chapters~\ref{chap:soc-net}-\ref{chap:stc-net} showed that a meaningful outcome metric can be attached to the scope of a build, which in our case is the build outcome as perceived by the developer.
We postulate that this can be done for other scopes as well.
Consider for instance the scope of a software change.
The work by Slivserski et al~\cite{sliwerski:notes} showed a way to infer whether a software change can be induced a fix worthy failure and thus yielding a similar outcome metric as build success for software changes. 

\paragraph{Social Networks}
In Chapter~\ref{chap:soc-net} we reported on a study~\cite{wolf:icse:2009} using social networks related to software builds to predict build outcome.
The construction of social networks depends mainly on the possibility of tracing communication artifacts such as work item comments to the respective build.
In our case studies with the IBM Rational Team Concert development team as well as the course study presented in Chapter~\ref{chap:actionable} the development teams used systems that made links between software builds explicit through their issue tracking and version control systems.
To apply this step in project that do not make these links between artifacts explicit methods proposed through the Hipicat~\cite{cubranic:tse:2005} by Cubranic et al can be used to heuristically infer links between artifacts of interest.

The nature of how communication recorded electronically enables the automatic construction of social networks as soon as the communication artifacts are recorded within the respective system.
In the case of modern issue trackers that would mean, as soon as a developer adds a comment to a work item this information can be used to complement all social networks this comment is relevant to.

\paragraph{Technical Networks}
In the beginning of this thesis in Chapter~\ref{chap:tech-net} we detailed related work that looked into leveraging technical networks to predict software failures.
Chapter~\ref{chap:tech-net} presents work that focuses on files or binary/package level predictions and thus build technical networks around a software artifact that is hard to associate to a specific time in the project.
Most of the research reported on resolves this by looking at a milestone level and thus declaring the milestone as the scope and relate parts of the milestone level technical network to software artifacts in whose failure-proneness is of interest.

In our study described in Chapter~\ref{chap:actionable}, we showed how to use the method proposed by Blincoe et al~\cite{blincoe:cscw:2012} to construct technical networks while developer are exploring and modifying source code.
In contrast to constructing social networks once changes to the version control are committed the work or task associated with the change has been finished.
Thus unless the recommendations help after the actual work has been accomplished there is an inherent need to build technical networks as the code is modified for a given task.
For example, when looking into increasing productivity generating recommendations from technical networks after the work has been accomplish is of little use.

\paragraph{Insights}
Once the social and technical networks are created the combined socio-technical networks can then be contrasted based on the outcome metric each network is associated with.
In Chapter~\ref{chap:stc-net} and~\ref{chap:actionable} we showed how to generate insights using a binary outcome metric by decomposing the networks and measuring the correlation of the network parts with the outcome metric, thus yielding insights that can be used to directly change the social network to be more liekly to yield the desired outcome.

Although we showed through the studies reported in this thesis that there is ample evidence supporting the usefulness of the approach to leverage socio-technical congruence to generate actionable knowledge, there are limitations that we discuss in the next section.

\section{Threats to Validity}
Although we discussed threats to validity in each chapter that presented research that we conducted in the course of this thesis we will address the overall threat to the approach of leveraging socio-technical networks to generate actionable knowledge as described in the previous sections.
We identified four threats that need to be mentioned, two are concerned with the limited data available to conduct our studies whereas the other two are concerned with the limiting nature of providing statistical inferences.

% limited number of studies
In this thesis we the studies we presented draw information from observational studies and studies relying on the same development repository.
Although this limits the generalizability of the findings presented as well as the validity of the inferred approach, we think that the approach still holds merit as the studies that lie the foundation for the validity of generating insights in real time are derived from and industrial project comprising more than one hundred developers at a large software corporation.
This in-depth relationship creating by working together with the IBM Rational Team Concert development team limits the amount of data available for the studies we presented but this in-depth relationship enables us to better interpret the collected data as well as gain a deeper understanding of the organization and their processes and how they influence the data.
In the case of the in class study, we aimed to minimize the conclusions we drew to only serve as a feasibility study to demonstrate that technical networks can be constructed in real time as well as give some evidence that potential recommendations could have prevented build failures from occurring.

% never traced found patterns to issues/rellied on statistical analysis
Chapters~\ref{chap:stc-net2} and~\ref{chap:stc-net} demonstrated that constructing the socio-technical networks is feasible and in Chapter~\ref{chap:stc-net} we showed that there is a statistical relationship between the network configuration and build success that can be used to generate recommendations.
One issue that we will need to address in future work is showing a definite link between the insights presented in Chapter~\ref{chap:stc-net} and the actual build failures and to what extend the recommendations actually can prevent build failures from happening.
To mitigate this threat we showed some initial evidence of tracing a failed build back to its original failure source and showed that the failure could have been prevented with the socio-technical information available at the point in time when the error was introduced into the code base.

% approach not tested in the field
The final threat to the approach, which is related to the previously mentioned lack of tracing the basis of the recommendations back to actual build failures, is that we did not test it in the field to see how the recommendation potentially affect the development process.
Although the study we presented in Chapter~\ref{chap:talk} explored if the recommendations are made at an appropriate level of granularity as well as feedback to the usefulness of such recommendations.
Furthermore, the study conducted in a class room setting also suggests that there is value in generating such recommendations.

\section{FSE PAPER - Practical Implications}
\label{sec:implications}
Our findings have several implications for the design of collaborative systems.
By automating the analyses presented here we can incorporate the knowledge about
developer pairs that tend to be failure related in a real-time recommender
system. Not only do we provide the recommendations that matter to the upcoming
build, we also provide incentives to motivate developers to talk about their
technical dependencies. 

Such a recommender system can use project historical data to
calculate the likelihood that an upcoming build fails given a particular
developer pair that worked on that build without communicating to each other. In
the case of the pair (Adam, Bart) the system may recommend that these
developers should communicate about their technical dependencies, as there
would be a probability
of 91\% of failure of the next build should they not follow the
recommendation. This probability of failure serves as mechanism to
rank importance of a socio-technical gap and more importantly as an
incentive to act upon.

For management, such a recommender system can provide details about the
individual developers in, and properties of, these potentially problematic
developer pairs. Individual developers may be an explanation for the behaviour of
the pairs we found in Rational Team Concert. This may indicate developers that are
harder to work with or too busy to coordinate appropriately, prompting management
to reorganize teams and workloads. This would minimize the likelihood of a build
to fail, by removing the underlying cause of a pair to be failure related.
Similarly, as another example from our study, most developer pairs
consisted of developers that were part of different teams. In such
situations management may decide to investigate reasons for coordination
problems that include factors such as geographical or functional distance in the project.




\section{Conclusions and Future Work}
In this thesis we illustrated an approach to leverage the concept of social-technical congruence to generate actionable knowledge.
This five step approach focuses on defining two key parameters up front: (1) the scope of interest and (2) the outcome metric of interest.
The first parameter, scope, helps with constructing the socio-technical networks, with constructing the social networks being the third and constructing the technical networks being the fourth step, by supporting the selection of the best data sources.
the outcome metric guides the analysis to produce actionable knowledge in the form of indicators that positively or negatively influence the outcome metric (step 5). 

We derived this approach through a number of case studies that investigate the usefulness of social and socio-technical networks with respect to supporting developers improve software quality (Chapters~\ref{chap:soc-net}-\ref{chap:stc-net}).
Through these studies this approach crystallized to the point that in Chapter~\ref{chap:stc-net} we followed it ourselves.
The studies we conducted in the subsequent Chapters~\ref{chap:talk} and~\ref{chap:actionable} further explores the usefulness of the information with respect to whether experts expect the level of recommendations to be of use as well if these recommendations could be produced in real time and potentially preevnt issues from arising.

Each study by itself contributed the the overall body of knowledge besides furthering the goal of the thesis to lay the foundations of a recommendations system leveraging the construct of socio-technical congruence.
With our first study we gave empirical evidence of communication among software developers influencing software quality.
Although by itself not surprising that issues in communication can hinder productivity and introduce ambiguities that might lead to problems with respect to software quality, it is, to the best of our knowledge, the first study that instead of looking into content of individual  conversations takes a higher level approach and relates communication structures to software quality.

% stc and build success
The relationship between communication structure and build failures however significant has only a small effect on the overall success rate of software builds, the outcome metric we studied.
This lead us to include information about the system by adding technical dependencies as expressed by the source code among software developers.
Backed up y findings in the research area of socio-technical congruence we hypothesized that the technical relationships help to zero in onto the important relationships among developers that relate to build failures.
As the relationship between socio-technical congruence and productivity suggested influence on software quality, we showed in Chapter~\ref{chap:soc-net} that it actually predicts build failures with varying accuracy depending on the type of build.

% failure inducing pairs
Being able to predict whether a build fails already help developer to plan ahead with respect to future work, such as stabilizing the system in contrast to working on new features, but ultimately we want to be able to prevent builds from failing.
To that purpose we would need to influence the socio-technical network such that it takes a structure that is more favourable to build success.
We found that certain constellations within a socio-technical network, to be more precise pairings of software developer and their respective relationship, seem to be correlating with build success.
This evidence can be used to recommend action before the build is commenced in the sense that developer can investigate their relationship by for example discussing the code changes that created a technical relationship between them.

% talk or not to talk
Recommendations to manipulate the socio-technical network to improve the odds of a successful build is a good start, however the developers that need to follow these recommendations need to accepts them first.
It turns out, that developers are generally open to recommendations on a low level, such as on a change-set basis, but it depends on external factors such as development process.
For instance, we found depending on how close a development team to a software release is the more they focus on the implications of individual changes, whereas developers forcus more high level reusability issues at the beginning of a release cycle.

% leveraging stc in real time
Knowing that socio-technical congruence lends it self to produce actionable knowledge that has an acceptable form to support developers in the wild is one step short is still one step short in exploring the feasibility of proving to of use.
The final peace as we showed in Chapter~\ref{chap:actionable} is to show the feasibility of generating the recommendations at the right time.
Thus, we showed that socio-technical congruence can be used in real time to create actionable knowledge that might be of use to developers.

% future work
The work presented in this thesis lend it self to several obvious venues of future work, such as building and testing the recommendation system with several software development teams to study its impact.
A more interesting avenue to pursue is to explore what software architecture can support what kind of communication and organizational structure.
So far, the research around socio-technical congruence is pointing into the direction of changing how software developers coordinate their work, but returning to the original observation Conway made in that the software architecture will change to accommodate the communication structures inhering in an organization.
Therefore, analyzing software architectures with respect to the project properties, such as distribution of the development team or the organizational hierarchy, might yield valuable insight in guiding design decisions of the software product that not only take into account properties to increase the feature richness or maintainability of the software product but to properties of the organization and the development team to increase productivity to develop and quality of software product. 
