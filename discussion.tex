\startchapter{Discussion}
In this chapter we will discuss the contributions of this thesis.
We first present a detailed summary of the approach that this thesis described to leverage the potential hidden in the conceptulization of socio-technical congruence.
Then we follow up with a more detailed explanations on how the research presented in this thesis as well as other related work supports the validity of this approach.
Closing with some remarks on the general threats to validity.

\section{Approach to Leveraging Socio-Technical Congruence}
The main contribution of this thesis lies with the implicitly described approach to creating actionable knowlegde from the socio-technical congruence concept as Cataldo et al~\cite{} described it in their seminal paper.
Since the research questions we forumlated for this thesis only implicitly describe the approach, it is necessary to describe the approach to wrest actionable insights from the concept of socio-technical congruence.

Our approach took five steps:
\begin{enumerate}
\item Define scope of interest.
\item Define outcome metric.
\item Build social networks according to the scope in real time.
\item Build technical networks according to the scope in real time.
\item Generate actionable insights.
\end{enumerate}

% 1. Define scope of interest.
\paragraph{Scope} 
Each network, social, technical and thus socio-technical, needs to build with a specific scope in mind.
For example, throughout this thesis we used software builds a focus.
This focus defines the source and communication artifacts that are used to construct the socio-technical networks. 

% 2. Define outcome metric.
\paragraph{Outcome}
In order to derive useful insights from the constructed networks the scope used to construct them needs to be complimented with an outcome metric.
For example, throughout this thesis we used build success, a binary varibale that states whether a build is of acceptable quality.
With an outcome measure at hand we can contrast networks to gain insights.

% 3. Build social networks according to the scope in real time.
\paragaph{Social Networks}
After the defining the scope and outcome, the next step is to construct the socio-technical networks.
In our approach we first focus on the construction of the social part of the networks.
This involves identifying all communication channels that are programatically accessible in real time.
Some examples of communication channels that can be tapped into in real time are emails, forum style discussions, or text chats.
Gathering all to the selected scope relevant communication artifacts than yields the social-network.

% 4. Build technical networks according to the scope in real time.
\paragraph{Technical Networks}
To complement the social networks and thus create socio-technical networks we need to produce technical networks.
The main issues is not to rely on information that is only available after the work has been completed and been submitted to a version repository, but to gather information to construct networks in real time.
For instance, Blincoe et al~\cite{} proposed to use Mylyn\footnote{} events to accomplish the extraction of interactions with source code in real time.

% 5. Generate actionable insights.
\paragraph{Inishgts.}


\section{Grounding the Approach}
In this section we will discuss on how the approach described in the previous Section~\ref{} is grounded in the research presented in this thesis.
We will discuss and ground the feasability of each step of the approach in the research presented in this thesis and conclude this section with a more general remark on the evidence shwoing the value in the approach as a whole.

\pargrpah{Scope}
In Chapters~\ref{}-\ref{} we presented work that uses a specific focus namely the build.
Selecting a software build as the focus of building and generting insights into the socio-technical networks constructed for a given build.
With a software build representing a high level software artifact identifying this artifact is straight forward as the studies presented in this thesis sufficiently demostrated.
Generally any scope that is easily captured within a software artifact such as a software change or a work item can server as a scope sused to construct and generate insight from socio-technical networks.

% 2. Define outcome metric.
\paragraph{Outcome}
With our focus on software quality in the form of build success, we chose as outcome metric whether in the opinion of the developer a build was counted as success or not.
The studies we performed and reported on in Chapters~\ref{}-\ref{} showed that a meaninful outcome metric can be attached to the scope of a build, which in our case is the build outcome as percieved by the develope.
We postulate that this can be done for other scopes as well.
Consider for instance the scope of a software change.
The work by Slivserski et al~\cite{} showed a way to infer whether a software change can be induced a fix worthy failure and thus yielding a similar outcome metric as build success for software changes. 

% 3. Build social networks according to the scope in real time.
\paragraph{Social Networks}
In Chapter~\ref{} we reported on a study~\cite{wolf:icse:2009} using social networks related to software builds to predict build outcome.
The construction of social networks depends mainly on the possibility of tracing communication artifacts such as work item comments to the respective build.
In our case studies with the IBM Rational Team Concert development team as well as the course study presented in Chapter~\ref{} the development teams used systems that made links between software builds explicit through their issue tracking and version control systems.
To apply this step in project that do not make these links between artifacts explicit methods proposed through the Hipicat~\cite{} by ... et al can be used to heuristcally infer links between artifacts of interest.

The nature of how communication recorded electronically enables the automatic construction of social networks as soon as the communication artifacts are recorded within the respective system.
In the case of modern issue trackers that would mean, as soon as a developer adds a comment to a work item this information can be used to complement all social networks this comment is relevant to.

% 4. Build technical networks according to the scope in real time.
\paragraph{Technical Networks}
In the beginning of this thesis in Chapter~\ref{} we detailed related work that looked into leveraging technical networks to predict software failures.
Chapter~\ref{} presents work that focuses on files or binray/package level predictions and thus build technical networks around a software artifact that is hard to associate to a specific time in the project.
Most of the research reported on resolves this by looking at a milestone level and thus declaring the milestone as the scope and relate parts of the milestone level technical network to software artifacts in whose failure-proness is of interest.

In our study described in Chapter~\ref{}, we showed how to use the method proposed by Blincoe et al~\cite{} to construct technical networks while developer are expoloring and modifying source code.
In contrast to constructing social networks once changes to the version control are committed the work or task associated with the change has been finished.
Thus unless the recommendations help after the actual work has been accomplished there is an inherent need to build technical networks as the code is modified for a given task.
For example, when looking into increasing productivity generating recommendations from technical networks after the work has been accomplish is of little use.

% 5. Generate actionable insights.
\paragraph{Insights}
Once the social and technical networks are created the combined socio-technical networks can then be contrasted based on the outcome metric each network is associated with.
In Chapter~\ref{} and~\ref{} we showed how to generate insights using a binary outcome metric by decomposing the networks and measuring the correlation of the network parts with the outcome metric, thus yielding insights that can be used to directly change the social network to be more liekly to yield the desired outcome.

% concluding sentence leading into the threats to validity and maybe a sentence on why we will omit practical and research implications

\section{Threats to Validity}
