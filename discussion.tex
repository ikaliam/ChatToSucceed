\startchapter{Discussion}
In this chapter we will discuss the contributions of this thesis.
We first present a detailed summary of the approach that this thesis described to leverage the potential hidden in the conceptulization of socio-technical congruence.
Then we follow up with a more detailed explanations on how the research presented in this thesis as well as other related work supports the validity of this approach.
Closing with some remarks on the general threats to validity.

\section{Approach to Leveraging Socio-Technical Congruence}
The main contribution of this thesis lies with the implicitly described approach to creating actionable knowlegde from the socio-technical congruence concept as Cataldo et al~\cite{} described it in their seminal paper.
Since the research questions we forumlated for this thesis only implicitly describe the approach, it is necessary to describe the approach to wrest actionable insights from the concept of socio-technical congruence.

Our approach took five steps:
\begin{enumerate}
\item Define scope of interest.
\item Define outcome metric.
\item Build social networks according to the scope in real time.
\item Build technical networks according to the scope in real time.
\item Generate actionable insights.
\end{enumerate}

% 1. Define scope of interest.
% 2. Define outcome metric.
% 3. Build social networks according to the scope in real time.
% 4. Build technical networks according to the scope in real time.
% 5. Generate actionable insights.


\section{Grounding the Approach}
\section{Threats to Validity}
