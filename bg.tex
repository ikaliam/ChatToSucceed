\startchapter{Background}
In this chapter we will give a rough overview over the challenge that we described in the introduction and how other work tried partially or in full to address these challenges.
Note that we will not go into great detail of the indvidual steps that we undertook in this thesis as we discuss related work in the appropriate chapter.
Thus we will discuss two areas: (1) the research around the concept of socio-technical congruence and (2) recommender systems in sofftware engineering.

\section{Socio-Technical Congruence}
As mentioned earlier this thesis explores to what extend cane we leverage the concept of socio-technical congruence, before we start discussion the work that has been done with respect to using the concept of socio-technical congruence to analyise softare development teams and their performance, we explain the socio-technical congruence conpect and it different ways it was implemented so far.
We close this section by discussion the next step we are going to take with respect to concept.

\subsection{Socio-Technical Congruence Definitions}
The literature ecploring and using the concept of socio-technical congruence often relies on two interconnected definitions of socio-technical congruence.
Originally as first defined by Cataldo et al~\cite{} socio-techncial congruence was a single metric describing how much of the work dependencies between developersa are covered by the commonuication between those developers, but the interest in socio-technical congruence took a broader view and instead of focusing on the metric researchers~\cite{} started focusing on the underlying construct conceptualizing the different connections among dvelopers.
Following we discuss the two commonly used approached to infer socio-technical dependecies among developers, starting with the traditional definition initially presented by Cataldo et al~\cite{} followed by a more network centric definition.

\subsubsection{Task Assignment and Dependecy}
Cataldo et al~\cite{} defined the technical dependencies among developers as the matrix multiplication of the matrix defining the assignment of a developer to a task with the matrix defining the dependencies among tasks multiplid with the invers of the matri defining the assigment of a devloper to a task.
Thus two matrices need to be infered from a data set: (1) task assignment matrix describing which developer is assigned to what task and (2) the task dependency matix describing which tasks share dependencies.

\paragraph{Task Assignment Matrix}
The task assginment matrix dimension is the number of developers times the number of tasks.
Each entry in the matrix denotes whether a given developer is assigned to a given task, note that this notation allows for more than one developer to be assigned to a task as well as one developer being assigned to multiple tasks.
This infromation is inferred from task manahement systems such as BugZilla\footnote{} or Jira\footnote{} that show who is assigned to work on a given task.
Thus constructing the task assigment matrix together with identifying developer and task is straight forward when a task management is available.

\paragraph{Task Dependency Matrix}
The task dependecy matrix dimension is the number of tasks times the number of tasks with each row and column representing all tasks.
Each entry in the task dependecy matrix indicated whether two tasks have a dependecy, note that although this matrix allows for defining a direction with respect to the dependency but as Cataldo et al~\cite{} forumulated it and other researchers used it~\cite{} the direction is essentially not used.
The task dependency matrix is populated by identifying the code written to finish a task and infer dependencies among the various code changes implementing different tasks.
For example, Cataldo et al~\cite{} defined two tasks to be dependend if the associated changes modify the same file. 

\begin{figure}[ht]
\centering
\includegrahics{}
\caption{Calculating technical dependecies among developer using the task assignment nd task dependency matrix.}
\label{chap:3:fig:example:stc:cataldo}
\end{figure}
The final calculation of the technical dependency among developer follows the formula presented below:

\begin{equation}
something
\end{equation}

Figure~\ref{chap:3:fig:example:stc:cataldo} shows and example on how to derive the technical dependencies among developers given a task assignment and task dependency matrix.
Following the formula presented in Equation~\ref{}, we multiply the task assignment matrix with the task dependecy matrix with the transposed task assignment matrix to optain a matrix of dimension of number of developers by number of developer with each entry in the matrix greater than 0 denoting a technical dependency between two developers.
For instance developer Adam and Eve have a resulting dependency because they are both assigned to tasks that have a dependency in the task dependecy matrix. 
This resluting matrix is also reffered to as the coordination needs matrix.

The technical dependecy matrix opteined throught the matrix multiplication described, needs to be contrasted with the actual coordintation that happened in the project.
For this purpose Cataldo et a~\cite{} proposed to create a matrix recording whether two developer coordinate their work.
Communication is often~\cite{} used as a proxy for coordination and as soon as there is an instance of coordination among two developers recorded, such as emails or posting together comments on the same task.
To congruence metric itself is the ratio between developers that have both a technical dependency and did coordinate over the number of developer that have a technical dependency.

% lead into the next subsection
The actual coordination matrix depics a social network with developer being nodes and coordination instances edges.
Similarly the coordination needs matrix depicts a social network connecting developers when they share a technical dependency.
Thus another method to approach socio-technical congruence instead theough the explict definition of the task assignment and the task depenecy matrix is to take a more social networks analysis point of view and construct the two types of social networks directly as we discuss in the next section.

\subsubsection{Social and Technical Networks}

\subsection{Socio-Technical Congruence and Performance}
\subsection{Socio-Technical Congruence Next Steps}

\section{Recommender Systems in Software Engineering}
