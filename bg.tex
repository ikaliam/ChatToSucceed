% !TEX root = thesis.tex
\startchapter{Background}
\label{chap:bg}
In this chapter we will give a rough overview over the challenge that we described in the introduction and how other work tried partially or in full to address these challenges.
Note that we will not go into great detail of the individual steps that we undertook in this thesis as we discuss related work in the appropriate chapter.
Thus we will discuss two areas: (1) the research around the concept of socio-technical congruence and (2) recommender systems in software engineering.

\section{Socio-Technical Congruence}
As mentioned earlier this thesis explores to what extend cane we leverage the concept of socio-technical congruence, before we start discussion the work that has been done with respect to using the concept of socio-technical congruence to analyze software development teams and their performance, we explain the socio-technical congruence concept and it different ways it was implemented so far.
We close this section by discussion the next step we are going to take with respect to concept.

\subsection{Socio-Technical Congruence Definitions}
The literature exploring and using the concept of socio-technical congruence often relies on two interconnected definitions of socio-technical congruence.
Originally as first defined by Cataldo et al~\cite{cataldo:cscw:2006} socio-technical congruence was a single metric describing how much of the work dependencies between developers are covered by the communication between those developers, but the interest in socio-technical congruence took a broader view and instead of focusing on the metric the focus shifted to the underlying construct conceptualizing the different connections among developers.
Following we discuss the two commonly used approached to infer socio-technical dependencies among developers, starting with the traditional definition initially presented by Cataldo et al~\cite{cataldo:cscw:2006} followed by a more network centric definition.

\subsubsection{Task Assignment and Dependecy}
Cataldo et al~\cite{cataldo:cscw:2006} defined the technical dependencies among developers as the matrix multiplication of the matrix defining the assignment of a developer to a task with the matrix defining the dependencies among tasks multiplied with the inverse of the matrix defining the assignment of a developer to a task.
Thus two matrices need to be inferred from a data set: (1) task assignment matrix describing which developer is assigned to what task and (2) the task dependency matrix describing which tasks share dependencies.

\paragraph{Task Assignment Matrix}
The task assignment matrix dimension is the number of developers times the number of tasks.
Each entry in the matrix denotes whether a given developer is assigned to a given task, note that this notation allows for more than one developer to be assigned to a task as well as one developer being assigned to multiple tasks.
This information is inferred from task management systems such as BugZilla\footnote{\url{http://www.bugzilla.org}} or Jira\footnote{\url{http://www.atlassian.com/software/jira}} that show who is assigned to work on a given task.
Thus constructing the task assignment matrix together with identifying developer and task is straight forward when a task management is available.

\paragraph{Task Dependency Matrix}
The task dependency matrix dimension is the number of tasks times the number of tasks with each row and column representing all tasks.
Each entry in the task dependency matrix indicated whether two tasks have a dependency, note that non-zero entries refer to the existence of a dependency but not its strength.
The task dependency matrix is populated by identifying the code written to finish a task and infer dependencies among the various code changes implementing different tasks.
For example, Cataldo et al~\cite{cataldo:cscw:2006} defined two tasks to be dependent if the associated changes modify the same file. 

\begin{figure}[ht]
\centering
\[
\left(
\begin{matrix}
1 & 0 & 1 & 1\\
0 & 0 & 0 & 1\\
1 & 0 & 0 & 0\\
0 & 1 & 0 & 1
\end{matrix}
\right)
\times
\left(
\begin{matrix}
0 & 1 & 0 & 0\\
1 & 0 & 1 & 0\\
0 & 1 & 0 & 1\\
0 & 0 & 1 & 0
\end{matrix}
\right)
\times
\left(
\begin{matrix}
1 & 0 & 1 & 1\\
0 & 1 & 0 & 1\\
1 & 0 & 1 & 0\\
1 & 1 & 0 & 1
\end{matrix}
\right)
=
\left(
\begin{matrix}
0 & 1 & 0 & 3\\
1 & 0 & 0 & 0\\
0 & 0 & 0 & 1\\
3 & 0 & 1 & 2
\end{matrix}
\right)
\]
\caption{Calculating technical dependencies among developer using the task assignment and task dependency matrix.}
\label{chap:3:fig:example:stc:cataldo}
\end{figure}
The final calculation of the technical dependency among developer follows the formula presented below:

\begin{equation}
\label{eq:stc:cataldo}
\text{Task Assignment} \times \text{Task Dependency} \times \text{Task Assignment}^{\text{T}} = \text{Coordination Needs}
\end{equation}

Figure~\ref{chap:3:fig:example:stc:cataldo} shows and example on how to derive the technical dependencies among developers given a task assignment and task dependency matrix.
Following the formula presented in Equation~\ref{eq:stc:cataldo}, we multiply the task assignment matrix with the task dependency matrix with the transposed task assignment matrix to obtain a matrix of dimension of number of developers by number of developer with each entry in the matrix greater than 0 denoting a technical dependency between two developers.
For instance developer Adam and Eve have a resulting dependency because they are both assigned to tasks that have a dependency in the task dependency matrix. 
This resulting matrix is also referred to as the coordination needs matrix.

The technical dependency matrix obtained through the matrix multiplication described, needs to be contrasted with the actual coordination that happened in the project.
For this purpose Cataldo et a~\cite{cataldo:cscw:2006} proposed to create a matrix recording whether two developer coordinate their work.
Communication is often~\cite{cataldo:cscw:2006,kwan:tse:2011,valetto:msr:2007,ducheneaut:cscw:2005,ehrlich:stc:2008,wolf:icse:2009} used as a proxy for coordination and as soon as there is an instance of coordination among two developers recorded, such as emails or posting together comments on the same task.
To congruence metric itself is the ratio between developers that have both a technical dependency and did coordinate over the number of developer that have a technical dependency.

The actual coordination matrix depicts a social network with developer being nodes and coordination instances edges.
Similarly the coordination needs matrix depicts a social network connecting developers when they share a technical dependency.
Thus another method to approach socio-technical congruence instead through the explicit definition of the task assignment and the task dependency matrix is to take a more social networks analysis point of view and construct the two types of social networks directly as we discuss in the next section.

\subsubsection{Social and Technical Networks}
Since the task dependency as we saw earlier depends on the changes made to the software and their dependencies through the code it is often easier to directly construct the coordination needs matrix or for that matter the social network connecting developer via technical dependencies form the changes made to the system.
This is possible since changes to a software system are usually recorded in a source code repository and each change belongs to a developer.
Thus research~\cite{cataldo:cscw:2006,kwan:tse:2011,valetto:msr:2007,ducheneaut:cscw:2005,ehrlich:stc:2008} working with the socio-technical congruence concept with a social network view contrast social and technical networks.

\paragraph{Technical Networks}
In Cataldo et al's~\cite{cataldo:cscw:2006} formulation of technical dependencies he is constructing them by multiplying the task assignment and task dependency matrix.
Since the task dependency matrix is inferred from the overlap in code modifications, say both tasks are accomplished by modifying the same source code files, the technical dependencies among developers can be directly inferred from a software repository.
This more direct approach enables for the construction of technical networks, connecting developers through the dependencies of the changes they made to a software project, enables us to construct technical networks without the need of accessing a task management system.

\paragraph{Social Networks}
The social network representation of the ongoing communication is exactly the same as the actual coordination matrix as described by Cataldo et al's~\cite{cataldo:cscw:2006} as the matrix is in effect a way to represent a network (also known as adjacency matrix).

The tricky part of this approach is to match the social and technical networks as the usernames used for code repositories and task management can be different, this is especially an issue with open source development as they are less governed by processes demanding naming conventions of account names~\cite{schroeter:isese:2006}.

\subsection{Socio-Technical Congruence and Performance}
Social-technical congruence as originally observed by Conway~\cite{conway:datamination:1968} stating that any product developed by an organization will inevitably structured as that organizations communication structure.
From this starting point Cataldo et al~\cite{cataldo:cscw:2006} as well as other researchers~\cite{valetto:msr:2007,ducheneaut:cscw:2005,ehrlich:stc:2008} investigated if there exists a connection between productivity and the overlap of the communication among developers and the technical dependencies of their work.
The communication is taken as the organizational communication structure and the technical dependencies between the work each developer needs to accomplish as the systems organization.
The idea is that if the communication structure completely contains the work dependencies among developers, that developer will be able to accomplish their tasks faster for reasons that are mainly due to knowledge seeking and sharing~\cite{desouza2006:knowledge}.
For example, a developer can better accomplish her task if she is talking directly to co-workers that need to modify related code to avoid failures or because someone can help her understand the impact the code she is about to modify better.

The main performance criteria research investigated to measure the effect of socio-technical congruence is the time it takes to complete a task.
For this purpose Cataldo et al~\cite{cataldo:cscw:2006} measures the congruence on a task basis and correlating congruence metric drived from the overlap of the social and technical network with the time it took to resolve the task.
Overall Cataldo et al~\cite{cataldo:cscw:2006} found and were confirmed by other studies~\cite{valetto:msr:2007,ehrlich:stc:2008} that there is a statistically significant relation between the amount of congruence and a tasks resolution time.

Considering that the main issue with the productivity is the longer way of communication through intermediates in contrast to the direct way lends itself to open the question of whether congruence can predict software quality.
The rational is that the additional complication in communication might introduce errors into the system due to awareness.
Meneely et al~\cite{meneely:fse:2008} showed that socio-technical networks can be used to predict software failures whereas Kwan et al~\cite{kwan:tse:2011} found that the congruence metric statistically relates to build success.
In both cases, software quality and productivity there are some additional steps that need to be taken to fully leverage the concept of socio-technical congruence.



\section{Source Code and Failure}
In this section we go over the evidence empirical research has produced that bring technical networks produced through source code dependencies into relation with software defects.

\subsection{Software Metrics}
\label{chap:6:measure}
We break down software metrics into three categories:
(1) code complexity metrics that measure the interdependencies between low level software artifacts,
(2) object oriented metrics often measure very localized aspects or the class interdependencies,
and (3) metrics that measure code interdependencies such as function fan in and fan out.

\subsubsection{Code Complexity Metrics}
Defect prediction heavily uses metrics the describe code to predict defect proneness of files and other levels of software artifacts.
%
The two most common complexity metrics used are Lines of Code in a software artifact and Mc Cabe's complexity metric~\cite{mccabe:ieee:1976}.
Both metrics have been found to exhibit a medium to strong correlation across various software projects.
In principal, source code files or methods with more code lines have a higher changes to describe a more complex data flow.
Similarly, both metric indicate that the more complex code is the more relations it has to other parts of the code.

The first study undertaken to show the relation between multiple code metrics including lines of code and other metric such as Mc Cabe's complexity as well as other metrics, such as fan in and fan out which we will cover in Section~\ref{chap:6:sub:depmet}, was undertaken by Nagappan et al~\cite{nagappan:icse:2006} at Microsoft.
Both lines of code and Mc Cabe's complexity showed a correlation to post-release failures per file, failures that are reported by the customer within the first six months after release, of medium strength.

Although we consider Nagappan at al work to be among the first comprehensive studies into predicting post-release defects using multiple source code metrics, other studies the relation ship of particular metrics with defect density.
For instance, Koru et al~\cite{koru:promise:2005} conducted an in-depth analysis of the relation between the number of lines of code within a file and the relationship to defect density.
Size not only is a defect predictor in many forms, e.g., average function size or file size, but the size of components also determines how well other predictors can work on it.

In 2005 one year before Nagappan et al published their study on the relation of code metrics to defect density, Nikora et al~\cite{nikora:metrics:2005} discussed the use of graph measures used on the control flow graph.
In the light of Mc Cabe's complexity metric that gives one value to the overall control flow graph within a software artifact, Nikora et al characterize the control flow graph using graph measures.

Complexity and size metrics are often used as a control variable to ensure that newly found metrics are not simply a new representation of size or complexity and actually add more predictive power to already known metrics.
Hence, there is ample evidence that lines of code~\cite{shihab:esem:2010,arisholm:isese:2006,jiang:promise:2008,knab:msr:2006,zhang:icsm:2009} and Mc Cabe's complexity~\cite{nagappan:icse:2006,shihab:fse:2011,zimmermann:fse:2009,jiang:promise:2008,zimmermann:promise:2007} have a moderate correlation to software defects on a file level.

Despite their moderate usefulness complexity measures as we know them so far especially lines of code and Mc Cabe's complexity metric do not generalize across projects.
This means, using those metrics as predictors adjusted to one project usually offers poor predictions when applied to another~\cite{zimmermann:fse:2009}. 


\subsubsection{Change Complexity Metrics}
Besides measuring the complexity of the source code artifacts, measuring the complexity of the development of those code artifacts can also indicate at dependencies across artifacts.
During the evolution or maintenance of a project files are continuously edited to extend or improve the exiting project.
These changes implicitly change the dependencies to other software artifacts by changing call and data dependencies and thus construct and measure an implicit software artifact network.
Furthermore, changing files can create dependencies among the owner of the modified or affected and thus construct a technical network.

Software artifact modifications are seldomly equally distributed across the project time and the artifacts related to a project.
Thus, software artifact changes lend themselves to be studies in relation to software quality.
The simplest metric to measure on a file level is to count the amount of change that happened to a file within a given time frame~\cite{li:metrics:2005,moser:icse:2008,cataldo:icse:2011}.

The next logical step after investigating the amount of change is to characterize the complexity of the changes made to a software artifact. 
These churn metric measure similarly to lines of code the distribution of the change size in a given time frame~\cite{nagappan:icse:2005,shihab:fse:2011,zimmermann:fse:2009,bell:promise:2011}.
To take it one step further compared to studying the actual change sizes per software artifact the change of the change sizes poses a good predictor on the grounds that changes that seem out of the ordinary are more likely to introduce issues~\cite{hassan:icse:2009}.
Instead of looking into high level artifacts investigating low level changes on the level of control flow instructions such as if-then-else clauses also yields a good failure predictor~\cite{giger:msr:2011}.

Similarly to the code complexity metrics, change complexity metrics are also not necessarily able to predict failure density for a project when trained with data from a different projects~\cite{zimmermann:fse:2009}.


\subsubsection{Object Oriented Metrics}
\label{chap:6:sub:oom}
Object oriented metrics are taking into account more language specific constructs that in the case of object oriented programming are often describing the relationship between object, such as inheritance depth.
Besides measuring the relationships between objects object oriented metrics also measure the complexity of objects using metrics such as method counts.
Method counts have been used by Nagappan et al~\cite{nagappan:icse:2006} and Arisholm et al~\cite{arisholm:isese:2006} showed a moderately strong correlation between the amount of methods within a class and both post-release failures in Windows Vista and the defect density of components in telecommunication software.
%
Direct measures of object relationship, such as inheritance depth~\cite{chidamber:tse:1994}, sub-classing~\cite{chidamber:tse:1994} and object coupling~\cite{chidamber:tse:1994}, can predict software quality~\cite{nagappan:icse:2006,arisholm:isese:2006,english:promise:2009}.



\subsubsection{Dependency Metrics}
\label{chap:6:sub:depmet}
Besides the three types of metrics discussed that measure complexity of code and in the case of object oriented metrics sometimes actual dependencies between software artifacts.
Metrics such as fan-in and fan-out~\cite{henry:tse:1981} of methods are directly measuring the dependencies between software artifacts.

Counting the number of dependencies either through metrics such as fan-in or fan-out or straight up counting call dependencies between software artifacts already yield moderate failure predictors~\cite{cataldo:icse:2011,nagappan:icse:2006,arisholm:isese:2006,knab:msr:2006,shin:msr:2009}.
Note that although most of the object oriented metrics we mentioned in the previous section (Section~\ref{chap:6:sub:oom}) are also dependency metrics we will not reiterate them in this section. 
Using more course grained dependency on the module level, or in other words counting dependencies across module boundaries, can also yield good failure predictors~\cite{jiang:promise:2008}.
Another dependency metric creating dependencies between files or larger modules, such as Java packages, uses the usage relationship between software artifacts as they can be defined at the time of a software projects design~\cite{schroeter:isese:2006,dualaekoko:esem:2009}.

Earlier we showed that the definition of complexity used on software artifacts such as source code files can also be extended to the notion of change complexity such as code churn, similarly dependencies between source code artifacts can be inferred using change information.
Zimmermann et al~\cite{zimmermann:icse:2004} and D'Ambros et al~\cite{dambros:wcre:2009} used the information about co-chaning files, that are files that are frequently changed togehter in the same change-set, to determine how dangerous it can be to not change those files together.
Zimmermann et al's work originally was intended to recommend with files to additionally change to an original software change  found that violating those co-change dependencies can lead to software failure.
Build on that work D'Ambros et al used the amount of those violations to build a prediction model that is able to predict defect likelihood for a file in a given project release. 


\subsection{Artifact Networks}
\label{chap:6:an}
This Section serves the purpose of exploring the existing work done on failure prediction that consciously leverages networks that are build from linked software artifacts.

Using dependencies within a product one can construct a network of software artifacts that share dependencies among each other.
Artifacts that have direct dependencies are sometimes in the case of source code referred to as code peers.
One interesting property of code peers is that in case a code peer exhibits a defect it increases the likelihood that the code artifact whose peer contains a defect has a higher likelihood to contain a defect by itself~\cite{nguyen:icse:2010}.

From this notion of a code peer and the influence a code peer or more generally an artifact peer can have on the quality of an artifact, analyzing these network with respect to an artifact and it surrounding artifacts.
In a first study Zimmermann et al~\cite{zimmermann:icse:2008} studied data and call dependency measures a single artifact has to its dependent artifacts and found it to be a good predictor for software defects.

In a follow up study Zimmermann et al~\cite{zimmermann:esem:2009} extended the influence of an artifacts peer by not solely focusing on an artifacts dependencies to its peers but taking into account the dependencies among an artifacts peers.
This enables the application of network measures and social-network measures to characterize this ego network constructed around a singe software artifacts.
As it turn out, the predictive power of such a network can be more powerful than only considering dependencies between an artifact and its peers~\cite{zimmermann:esem:2009}.

\subsection{Technical Networks}
\label{chap:6:tn}
To go from artifact network to technical network developers can be included in the already existing artifact network and thus be represented as a kind of artifact~\cite{pinzger:fse:2008}.
These two mode networks can be used to do the same analysis that Zimmermann et al~\cite{zimmermann:esem:2009,zimmermann:icse:2008} performed by focusing on the software artifacts to predict the failure likelihood of each.

Meneely et al~\cite{meneely:fse:2008} uses networks that consist of developers that within a given release modified the same file, which matches our definition of connecting developer as soon as artifacts they are related to are connected.
These networks, or rather social network measure extracted from these networks, are able to predict whether a file contain a failure.









\section{Recommendations in Software Engineering}
%Software engineering research produces several recommender systems.
%For us two types of recommender systems are of interest:
%(1) recommender systems that recommend code changes
%and (2) recommender systems that create awareness of change within the code.
%We are interested in these two types of recommender systems because they each directly influence the technical and social network, respectively.
%
%\subsection{Technical Recommendations}
%
%CHAPTER 6 (TECHNICAL NETWORKS) SHOULD GO HERE
%
%\subsection{Establishing Awareness}
Recommender systems often aim at providing developers and managers with awareness of technical relations among software developers.
Derived from the implication of socio-technical congruence described by Conway's Law~\cite{conway:datamination:1968} and later confirmed by Cataldo et al~\cite{cataldo:cscw:2006}, this additional awareness is mean to improve the coordination among software development especially in a distributed setting where coordination is most difficult~\cite{olson:hci:2000}.
In the following we describe five awareness systems, we are aware (pun intended) that this list is not exhaustive, but it presents a reasonable overview of awareness systems proposed by software engineering researchers that are recognized by the research community.

% ariadne
Ariadne~\cite{trainer2005:ariadne} provides awareness to developers by showing call dependencies between code a developer is working on and the code that she is potentially affecting when following call dependencies.
This allows a developer to see which developer she might need to coordinate her work with to not negatively impact the code of other developers that are using the code she is modifying.
%
% palantir
Palantir~\cite{sarma:cscw:2002} complements the dependencies among developer by providing the reverse awareness, namely, showing a developer what source code they are currently accessing in their workspace is affected by code changes submitted by co-workers.
For example, Palantir indicates which source code files have been changed in the mean time by other developer that are present in the developer current work space and thus might hint at possible merge conflicts.

% tesseract
Tesseract~\cite{arma:icse:2009} extends the concept of showing code dependencies among developers by fostering awareness through visualizing task and developer centric socio-technical networks, which would extend the technical networks directly and indirectly shown in Ariadne and Palantir by a social component.
A task centric socio-technical networks is build from all developers and source code changes that are related through code dependencies or task discussions to a specific task.
These task centric socio-technical networks are complemented by developer centric networks, that show for a specific developer what social, technical, or socio-technical relationships she has with her colleagues.

% proxi scentia
Ariadne, Plantir, and Tesseract suffer from the issue that they cannot provide real time feedback on changes in the technical networks, as they can only record changes made to source code as developer commit complete changes to the source code repository. 
Proxiscentia~\cite{borici:chase:2012} address this issue by implementing an apporach proposed by Blincoe et al~\cite{blincoe:cscw:2012} to instrument IDE's used by software developers and gather code edit events as recorded by tools such as Mylyn~\cite{kersten:aosd:2005}.
This enables a developer to be forewarned of changes that are made to related code as for example Palantir relies on or would enable Ariadne to take into account newly create source call dependencies as a developer is modifying code calling methods that are currently modified by another developer.

% Ensemble
All the previously described systems aim at providing developer with awareness that is meant to ease identifying colleagues to coordinate work with.
And as Tesseract shows being aware of the ongoing social connections is valuable as well, as it gives you an idea who one already coordinated with, it also gives you an idea on who can give advice related to work currently performed.
This opens the way for awareness systems that provide either notification systems about changes to any relevant artifact or enable developers to search across multiple repositories to find answers with respect to impact and expertise seeking.

For example, Ensemble~\cite{xiang:rsse:2008} provides a constant stream of events consisting of modifications to artifacts that are related to the stream owner.
If developer Adam posts a comment on a task owned by developer Eve, then Eve's stream of events would contain an event showing that Adam commented on her task.
Similarly, the stream of a developer also contains information about relevant code modifications that overlap or potentially interact with code he or she previously modified.

% code book
% hipicat
On the other hand systems such as HipikaT~\cite{cubranic:tse:2005} and Codebook~\cite{begel:icse:2010} provide a way to search and thus pull information about relationships between code artifacts and developer based on various sources such as emails, task managements systems, code repositories, or any other electronically accessible development repository.
In contrast to as system such as Essemble, search based systems will not flood developers with a large amount of notifications that are of little interest to developer, yet at the same time search based systems will not be able to notify developer if something of great importance that require immediate attention, such as a build that was meant as a release candidate failed.








\section{Next Steps/Research Questions}
The concept of socio-technical congruence shows potential to help make software development more efficient.
Cataldo et al~\cite{cataldo:cscw:2006} demonstrated its relation to productivity, and we show among other things in this thesis the ability to use socio-technical congruence to predict software quality.
The concept of socio-technical congruence lends itself to improve software development as it is based on social networks connecting developer on a coordination and technical level.
Because of the concept being based on networks it is possible to manipulate the networks.

There are two steps that need to be taken:
(1) identify the parts of the network that need to be changed
and (2) devise strategies to change the social network in a timely fashion.
In case of failure prevention the current way socio-technical congruence is constructed might be sufficient in that it is enough to compute after a task has been completed to assess the quality of the submitted code.
On the other hand, to contribute to the improvement of productivity we cannot afford to use networks constructed from submitted work as the work that we want to speed up has already been submitted.
Even though in the case of software quality it is sufficient to assess the work done to accomplish a task, providing earlier feedback is advantageous as it allows developer to intervene early on and spend less time on correct errors later on.

Any socio-technical network can be manipulated in two ways: (1) changing the technical dependencies among developer by refactoring or architectural changes to make them unnecessary and (2) by engaging developer in discussions about their recent work and therefore creating a coordination edge in the socio-technical network.
Since many product aren't developed from scratch and because architectural changes once development has been going on for a number of months are costly and time consuming~\cite{vangurp:jss:2002}, we aim at generating recommendations to change the actual coordination to improve the socio-technical network where it matters.
Therefore, as a first step we need to assess if the actual communication structure among software developers has an influence on software quality to lay the basis for manipulating the actual coordination to improve software quality.
As a follow up step, we need to explore the relationship between socio-technical networks and software quality.
Especially we are interested in the relationship in whether missing actual coordination in the face of a coordination needs is related to software failures.
Note, in this thesis we focus on software builds and their success as a measure for software quality.
Thus we formulate the following research questions:
\begin{description}
%
\item[RQ 1:] Does Socio-Technical Congruence influence build success?
  %
  \begin{description}
  \item[RQ 1.1:] Do Social Networks from repositories influence build success?
  \item[RQ 1.2:] Does Socio-Technical Networks influence build success?
  \end{description}
%
\end{description}
We explore \textbf{RQ~1.1} and \textbf{RQ~1.2} in Chapter~\ref{chap:soc-net} and Chapter~\ref{chap:stc-net2}, respectively, using data mining techniques as this allows us later to generate recommendations automatically.

Knowing that there is a relationship between the mis-match of coordination needs and actual coordination leads into the investigation of whether remedying this mismatch can be used to improve software quality.
Instead of suggesting that simply each coordination needs needs to be met, we want to focus on the most important violations that diminish software quality most often.
We start with investigations if single violations can be related to low software quality to form the basis for recommendations that can be ranked by their likelihood of having a negative influence on software quality.
Although knowing that socio-technical networks can be manipulated to increase software quality is a necessary step towards creating actionable knowledge, it is important that developers are willing to use the information and not to bother developer when they are not in need of recommendations.
Finally, to be able to deliver the recommendations we need a mechanism to compute socio-technical congruence in a timely manner.
Thus we formulate the following research questions:
\begin{description}
%
\item[RQ 2:] Can Socio-Technical Networks be used to recommend improvements? 
  %
  \begin{description}
  \item[RQ 2.1:] Can Socio-Technical-Networks be used to suggest improvements that matter?
  \item[RQ 2.2:] Does recommending on the lowest level make sense?
  \item[RQ 2.3:] When does recommending on the lowest level make sense?
  \item[RQ 2.4:] Can recommendation be given in a timely manner?
  \end{description}
\end{description}
Chapter~\ref{chap:stc-net} offers an insight into \textbf{RQ~2.1} by applying data mining techniques, whereas we investigate \textbf{RQ~2.2} and \textbf{RQ~2.3} through an in-depth participant observer study in Chapter~\ref{chap:talk}.
\textbf{RQ~2.4} is answered in Chapter~\ref{chap:actionable} in a qualitative way by exploring data gathered during a university course.









