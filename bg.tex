% !TEX root = thesis.tex
\startchapter{Background}
\label{chap:bg}
In this chapter we provide an overview of the challenges that we described in the introduction and how other work tried partially or in full to address these challenges.
Thus we will discuss three areas: (1) the research on software builds, (2) the research around the concept of socio-technical congruence, (3) failure prediction in software engineering, and (4) recommender systems in software engineering.

\section{Build Outcome}
Although software builds are important to delivering a software product as the final product is just the latest acceptable build, research in software builds focuses mainly on products and process that support the build process.
Software products supporting builds often intend to speed up the build process and the execution of all test cases to obtain an assessment of the quality of the build~\cite{maraia:book:2005}.
Similarly processes that focus on supporting software builds are predominantly dealing with issues of obtaining all required code changes from the different development teams and integrating this code into a final build as fast as possible without introducing additional issues.

The issues that shifts into focus once the actual process of creating the build is thoroughly optimized is to gain an idea of whether a build will fail or succeed.
If a project reaches a certain size, meaning the test suite grows considerably in size, the build process can take several hours just to run the whole test suite.
To determine whether developers need to stay in order to apply quick fixes such that the product can be shipped or handed over to a team starting heir work in a different time zone becomes important.

Research that aimed at discerning the success of a build leverages existing research on defect prediction to build a tree classifier to predict build outcome.
Hassan et al~\cite{hassan:ase:2006} found that time of the month and week are strong predictors of build success.
Note that we omitted the only two other studies to the best of our knowledge exploring predicting builds success because they form part of this thesis.

\section{Socio-Technical Congruence}
As mentioned earlier this thesis explores to what extent can we leverage the concept of socio-technical congruence. 
Before we start discussion the work that has been done with respect to using the concept of socio-technical congruence to analyze software development teams and their performance, we explain the socio-technical congruence concept.

\subsection{Socio-Technical Congruence Definitions}
The literature exploring and using the concept of socio-technical congruence often relies on two interconnected definitions of socio-technical congruence.
Originally defined by Cataldo et al~\cite{cataldo:cscw:2006} socio-technical congruence was a single metric describing how much of the work dependencies between developers are covered by the communication between those developers.
But the interest in socio-technical congruence took a broader view and instead of focusing on the metric the focus shifted to the underlying construct conceptualizing the different connections among developers.
Following we discuss the two commonly used approached to infer socio-technical dependencies among developers, starting with the traditional definition initially presented by Cataldo et al~\cite{cataldo:cscw:2006} followed by a more network centric definition.

\subsubsection{Task Assignment and Dependecy}
Cataldo et al~\cite{cataldo:cscw:2006} defined the technical dependencies among developers as the matrix multiplication of the matrix defining the assignment of a developer to a task with the matrix defining the dependencies among tasks multiplied with the inverse of the matrix defining the assignment of a developer to a task.
Thus two matrices need to be inferred from a data set: (1) task assignment matrix describing which developer is assigned to what task and (2) the task dependency matrix describing which tasks share dependencies.

\paragraph{Task Assignment Matrix}
The task assignment matrix dimension is the number of developers times the number of tasks.
Each entry in the matrix denotes whether a given developer is assigned to a given task, note that this notation allows for more than one developer to be assigned to a task as well as one developer being assigned to multiple tasks.
This information is inferred from task management systems such as BugZilla\footnote{\url{http://www.bugzilla.org}} or Jira\footnote{\url{http://www.atlassian.com/software/jira}} that show who is assigned to work on a given task.

\paragraph{Task Dependency Matrix}
The task dependency matrix dimension is the number of tasks times the number of tasks with each row and column representing all tasks.
Each entry in the task dependency matrix indicated whether two tasks have a dependency, note that non-zero entries refer to the existence of a dependency but not its strength.
The task dependency matrix is populated by identifying the code written to finish a task and infer dependencies among the various code changes implementing different tasks.
For example, Cataldo et al~\cite{cataldo:cscw:2006} defined two tasks to be dependent if the associated changes modify the same file. 

\begin{figure}[t!]
\centering
\[
\left(
\begin{matrix}
1 & 0 & 1 & 1\\
0 & 0 & 0 & 1\\
1 & 0 & 0 & 0\\
0 & 1 & 0 & 1
\end{matrix}
\right)
\times
\left(
\begin{matrix}
0 & 1 & 0 & 0\\
1 & 0 & 1 & 0\\
0 & 1 & 0 & 1\\
0 & 0 & 1 & 0
\end{matrix}
\right)
\times
\left(
\begin{matrix}
1 & 0 & 1 & 1\\
0 & 1 & 0 & 1\\
1 & 0 & 1 & 0\\
1 & 1 & 0 & 1
\end{matrix}
\right)
=
\left(
\begin{matrix}
0 & 1 & 0 & 3\\
1 & 0 & 0 & 0\\
0 & 0 & 0 & 1\\
3 & 0 & 1 & 2
\end{matrix}
\right)
\]
\caption{Calculating technical dependencies among developer using the task assignment and task dependency matrix.}
\label{chap:3:fig:example:stc:cataldo}
\end{figure}
The final calculation of the technical dependency among developer follows the formula presented below:
\begin{equation}
\label{eq:stc:cataldo}
\text{Task Assignment} \times \text{Task Dependency} \times \text{Task Assignment}^{\text{T}} = \text{Coordination Needs}
\end{equation}

Figure~\ref{chap:3:fig:example:stc:cataldo} shows an example on how to derive the technical dependencies among developers given a task assignment and task dependency matrix.
Following the formula presented in Equation~\ref{eq:stc:cataldo}, we multiply the task assignment matrix with the task dependency matrix with the transposed task assignment matrix to obtain a matrix of dimension of number of developers by number of developer with each entry in the matrix greater than 0 denoting a technical dependency between two developers.
For instance developer Adam and Eve have a resulting dependency because they are both assigned to tasks that have a dependency in the task dependency matrix. 
This resulting matrix is also referred to as the coordination needs matrix.

The technical dependency matrix obtained through the matrix multiplication described needs to be contrasted with the actual coordination that happened in the project.
For this purpose Cataldo et a~\cite{cataldo:cscw:2006} proposed to create a matrix recording whether two developers coordinate their work.
Communication is often~\cite{cataldo:cscw:2006,kwan:tse:2011,valetto:msr:2007,ducheneaut:cscw:2005,ehrlich:stc:2008,wolf:icse:2009} used as a proxy for coordination, such as emails or posting comments on task discussions in issue management systems.
The congruence metric itself is the ratio between developers that have both a technical dependency and did coordinate over the number of developers that have a technical dependency.

The actual coordination matrix depicts a social network with developer being nodes and coordination instances edges.
Similarly the coordination needs matrix depicts a social network connecting developers when they share a technical dependency.
Thus another method to approach socio-technical congruence instead through the explicit definition of the task assignment and the task dependency matrix is to take a more social networks analysis point of view and construct the two types of social networks directly as we discuss in the next section.

\subsubsection{Social and Technical Networks}
Since the task dependency as we saw earlier depends on the changes made to the software and their dependencies through the code it is often easier to directly construct the coordination needs matrix or for that matter the social network connecting developer via technical dependencies from the changes made to the system.
This is possible since changes to a software system are usually recorded in a source code repository and each change belongs to a developer.
Thus research~\cite{cataldo:cscw:2006,kwan:tse:2011,valetto:msr:2007,ducheneaut:cscw:2005,ehrlich:stc:2008} working with the socio-technical congruence concept with a social network view contrast social and technical networks.

\paragraph{Technical Networks}
In Cataldo et al's~\cite{cataldo:cscw:2006} formulation of technical dependencies they construct them by multiplying the task assignment and task dependency matrix.
Since the task dependency matrix is inferred from the overlap in code modifications, say both tasks are accomplished by modifying the same source code files, the technical dependencies among developers can be directly inferred from a software repository.
This more direct approach enables the construction of technical networks, connecting developers through the dependencies of the changes they made to a software project, without the need of accessing a task management system.

\paragraph{Social Networks}
The social network representation of the ongoing communication is exactly the same as the actual coordination matrix as described by Cataldo et al's~\cite{cataldo:cscw:2006} as the matrix is in effect a way to represent a network (also known as adjacency matrix).

The technical difficulties in this approach is to match the social and technical networks as the usernames used for code repositories and task management can be different, this is especially an issue with open source development as they are less governed by processes demanding naming conventions of account names~\cite{schroeter:isese:2006}.

\subsection{Socio-Technical Congruence and Performance}
Social-technical congruence as originally observed by Conway~\cite{conway:datamination:1968} states that any product developed by an organization will inevitably mirror the organization's communication structure.
From this starting point Cataldo et al~\cite{cataldo:cscw:2006} as well as other researchers~\cite{valetto:msr:2007,ducheneaut:cscw:2005,ehrlich:stc:2008} investigated whether the lack of this mirror relates to changes in productivity by investigating the overlap of communication among developers and their technical dependencies.
The communication among developers represents the organizational communication structure whereas the technical dependencies between the work each developer represents the products organization.
If the communication structure completely contains the work dependencies among developers, that developer will be able to accomplish their tasks faster for reasons that are mainly due to knowledge seeking and sharing~\cite{desouza2006:knowledge}.
For example, a developer can better accomplish her task if she is talking directly to co-workers that need to modify related code to avoid failures or because someone can help her understand the impact the code she is about to modify better.

The main performance criteria research investigated to measure the effect of socio-technical congruence is task completion time.
For this purpose Cataldo et al~\cite{cataldo:cscw:2006} measures the congruence on a task basis and correlating congruence metric derived from the overlap of the social and technical network with the time it took to resolve the task.
Overall Cataldo et al~\cite{cataldo:cscw:2006} found, and were confirmed by other studies~\cite{valetto:msr:2007,ehrlich:stc:2008}, that there is a statistically significant relation between the amount of congruence and a tasks resolution time.



\section{Source Code and Failure}
Technical dependencies that are used in the work of Cataldo et al~\cite{cataldo:cscw:2006} where studies before in the relation to software failures.
Because we are investigating how to improve communication among software developers following their technical dependencies among each other we give an overview over work that involves work on source code that directly or indirectly indicates technical dependencies. 

\subsection{Software Metrics}
\label{chap:6:measure}
We break down software metrics into three categories:
(1) code complexity metrics that measure the interdependencies between low level software artifacts,
(2) object oriented metrics often measure very localized aspects or the class interdependencies,
and (3) metrics that measure code interdependencies such as function fan in and fan out.

\subsubsection{Code Complexity Metrics}
Defect prediction heavily uses metrics the describe code to predict defect proneness of files and other levels of software artifacts.
%
The two most common complexity metrics used are Lines of Code in a software artifact and Mc Cabe's complexity metric~\cite{mccabe:ieee:1976}.
Both metrics have been found to exhibit a medium to strong correlation across various software projects.
In principal, source code files or methods with more code lines have a higher changes to describe a more complex data flow.
Similarly, both metric indicate that the more complex code is the more relations it has to other parts of the code.

A study undertaken to show the relation between multiple code metrics including lines of code and other metric such as Mc Cabe's complexity as well as other metrics, such as fan in and fan out which we will cover in Section~\ref{chap:6:sub:depmet}, was undertaken by Nagappan et al~\cite{nagappan:icse:2006} at Microsoft.
Both lines of code and Mc Cabe's complexity showed a correlation to post-release failures per file, failures that are reported by the customer within the first six months after release, of medium strength.

Although we consider Nagappan at al work to be among the first comprehensive studies into predicting post-release defects using multiple source code metrics, other studies the relation ship of particular metrics with defect density.
For instance, Koru et al~\cite{koru:promise:2005} conducted an in-depth analysis of the relation between the number of lines of code within a file and the relationship to defect density.
Size not only is a defect predictor in many forms, e.g., average function size or file size, but the size of components also determines how well other predictors can work on it.

In 2005 one year before Nagappan et al published their study on the relation of code metrics to defect density, Nikora et al~\cite{nikora:metrics:2005} discussed the use of graph measures used on the control flow graph.
In the light of Mc Cabe's complexity metric that gives one value to the overall control flow graph within a software artifact, Nikora et al characterize the control flow graph using graph measures.

Complexity and size metrics are often used as a control variable to ensure that newly found metrics are not simply a new representation of size or complexity and actually add more predictive power to already known metrics.
Hence, there is ample evidence that lines of code~\cite{shihab:esem:2010,arisholm:isese:2006,jiang:promise:2008,knab:msr:2006,zhang:icsm:2009} and Mc Cabe's complexity~\cite{nagappan:icse:2006,shihab:fse:2011,zimmermann:fse:2009,jiang:promise:2008,zimmermann:promise:2007} have a moderate correlation to software defects on a file level.

Despite their moderate usefulness complexity measures as we know them so far especially lines of code and Mc Cabe's complexity metric do not generalize across projects.
This means, using those metrics as predictors adjusted to one project usually offers poor predictions when applied to another~\cite{zimmermann:fse:2009}. 


\subsubsection{Change Complexity Metrics}
During the evolution or maintenance of a project files are continuously edited to extend or improve the exiting project.
Software artifact modifications are seldomly equally distributed across the project time and the artifacts related to a project.
Thus, software artifact changes lend themselves to be studies in relation to software quality.
The simplest metric to measure on a file level is to count the amount of change that happened to a file within a given time frame~\cite{li:metrics:2005,moser:icse:2008,cataldo:icse:2011}.

The next logical step after investigating the amount of change is to characterize the complexity of the changes made to a software artifact. 
These churn metric measure similarly to lines of code the distribution of the change size in a given time frame~\cite{nagappan:icse:2005,shihab:fse:2011,zimmermann:fse:2009,bell:promise:2011}.
Compared to studying the actual change sizes per software artifact the change of the change sizes poses a good predictor on the grounds that changes that seem out of the ordinary are more likely to introduce issues~\cite{hassan:icse:2009}.
Instead of looking into high level artifacts investigating low level changes on the level of control flow instructions such as if-then-else clauses also yields a good failure predictor~\cite{giger:msr:2011}.

Similarly to the code complexity metrics, change complexity metrics are also not necessarily able to predict failure density for a project when trained with data from a different projects~\cite{zimmermann:fse:2009}.


\subsubsection{Object Oriented Metrics}
\label{chap:6:sub:oom}
Object oriented metrics are taking into account more language specific constructs that in the case of object oriented programming are often describing the relationship between object, such as inheritance depth.
Besides measuring the relationships between objects object oriented metrics also measure the complexity of objects using metrics such as method counts.
Method counts have been used by Nagappan et al~\cite{nagappan:icse:2006} and Arisholm et al~\cite{arisholm:isese:2006} and they showed a moderately strong correlation between the amount of methods within a class and both post-release failures in Windows Vista and the defect density of components in telecommunication software.
%
Direct measures of object relationship, such as inheritance depth~\cite{chidamber:tse:1994}, sub-classing~\cite{chidamber:tse:1994} and object coupling~\cite{chidamber:tse:1994}, can predict software quality~\cite{nagappan:icse:2006,arisholm:isese:2006,english:promise:2009}.



\subsubsection{Dependency Metrics}
\label{chap:6:sub:depmet}
Besides the three types of metrics discussed that measure complexity of code and in the case of object oriented metrics sometimes actual dependencies between software artifacts.
Metrics such as fan-in and fan-out~\cite{henry:tse:1981} of methods are directly measuring the dependencies between software artifacts.

Counting the number of dependencies either through metrics such as fan-in or fan-out or straight up counting call dependencies between software artifacts already yield moderate failure predictors~\cite{cataldo:icse:2011,nagappan:icse:2006,arisholm:isese:2006,knab:msr:2006,shin:msr:2009}.
Note that although most of the object oriented metrics we mentioned in the previous section (Section~\ref{chap:6:sub:oom}) are also dependency metrics we will not reiterate them in this section. 
Using more course grained dependency on the module level, or in other words counting dependencies across module boundaries, can also yield good failure predictors~\cite{jiang:promise:2008}.
Another dependency metric creating dependencies between files or larger modules, such as Java packages, uses the usage relationship between software artifacts as they can be defined at the time of a software projects design~\cite{schroeter:isese:2006,dualaekoko:esem:2009}.

Earlier we showed that the definition of complexity used on software artifacts such as source code files can also be extended to the notion of change complexity such as code churn, similarly dependencies between source code artifacts can be inferred using change information.
Zimmermann et al~\cite{zimmermann:icse:2004} and D'Ambros et al~\cite{dambros:wcre:2009} used the information about co-changing files, files that are frequently changed together in the same change-set, to determine how dangerous it can be not to change those files together.
Zimmermann et al's work originally was intended to recommend which files to additionally change to an original software change found that violating those co-change dependencies can lead to software failure.


\subsection{Artifact Networks}
\label{chap:6:an}
Using dependencies within a product one can construct a network of software artifacts that are connected via those dependencies.
Artifacts that have direct dependencies are sometimes in the case of source code referred to as code peers.
One interesting property of code peers is that in case a code peer exhibits a defect it increases the likelihood that the code artifact whose peer contains a defect has a higher likelihood to contain a defect by itself~\cite{nguyen:icse:2010}.

From the notion of a code peer and its influence on other peers, can the idea of analyzing these network with respect to an artifact and it surrounding artifacts be derived.
In a first study Zimmermann et al~\cite{zimmermann:icse:2008} analyzed data and call dependency measures a single artifact has to its dependent artifacts and found it to be a good predictor for software defects.

In a follow up study Zimmermann et al~\cite{zimmermann:esem:2009} extended the influence of an artifacts peer by not solely focusing on an artifacts dependencies to its peers but taking into account the dependencies among an artifacts peers.
This enables the application of network measures and social-network measures to characterize this ego network constructed around a singe software artifacts.
As it turn out, the predictive power of such a network can be more powerful than only considering dependencies between an artifact and its peers~\cite{zimmermann:esem:2009}.

\subsection{Technical Networks}
\label{chap:6:tn}
To go from artifact network to technical network developers can be included in the already existing artifact network and thus be represented as a kind of artifact~\cite{pinzger:fse:2008}.
These two mode networks can be used for the same analysis that Zimmermann et al~\cite{zimmermann:esem:2009,zimmermann:icse:2008} performed by focusing on the software artifacts to predict the failure likelihood of each.

Meneely et al~\cite{meneely:fse:2008} uses networks that consist of developers that within a given release modified the same file.
These networks, or rather social network measure extracted from these networks, are able to predict whether a file contains a failure.









\section{Recommendations in Software Engineering}
In the software engineering community knowledge extracted from software repositories are usually brought to developers in the form or recommender systems.
Several recommender systems derived from the implication of socio-technical congruence described by Conway's Law~\cite{conway:datamination:1968} provide additional awareness to improve coordination among software development especially in a distributed setting where coordination is most difficult~\cite{olson:hci:2000}.
In the following we describe five such awareness systems.
We are aware that this list is not exhaustive. 
Nonetheless, we think this list presents a reasonable overview of awareness systems proposed by software engineering researchers.

% ariadne
\emph{Ariadne}~\cite{trainer2005:ariadne} provides awareness to developers by showing call dependencies between code a developer is working on and the code that she is potentially affecting when following call dependencies.
This allows a developer to see which developer she might need to coordinate her work with to not negatively impact the code of other developers that are using the code she is modifying.

% palantir
\emph{Palantir}~\cite{sarma:cscw:2002} complements the dependencies among developer by providing the reverse awareness, namely, showing a developer what source code they are currently accessing in their workspace is affected by code changes submitted by co-workers.
For example, Palantir indicates which source code files have been changed in the mean time by other developer that are present in the developer current work space and thus might hint at possible merge conflicts.

% tesseract
\emph{Tesseract}~\cite{sarma:icse:2009} extends the concept of showing code dependencies among developers by fostering awareness through visualizing task and developer centric socio-technical networks, which would extend the networks directly and indirectly shown in Ariadne and Palantir by a social component.
A task centric socio-technical networks is build from all developers and source code changes that are related through code dependencies or task discussions to a specific task.
These task centric socio-technical networks are complemented by developer centric networks, that show for a specific developer what social, technical, or socio-technical relationships she has with her colleagues.

% proxi scentia
Ariadne, Plantir, and Tesseract suffer from the issue that they cannot provide real time feedback on changes in the technical networks, as they can only record changes made to source code as developer commit complete changes to the source code repository. 
\emph{Proxiscentia}~\cite{borici:chase:2012} address this issue by implementing an approach proposed by Blincoe et al~\cite{blincoe:cscw:2012} to instrument IDE's used by software developers and gather code edit events as recorded by tools such as Mylyn~\cite{kersten:aosd:2005}.
This enables a developer to be forewarned of changes that are made to related code as for example Palantir relies on or would enable Ariadne to take into account newly create source call dependencies as a developer is modifying code calling methods that are currently modified by another developer.

% Ensemble
\emph{Ensemble}~\cite{xiang:rsse:2008} provides a constant stream of events consisting of modifications to artifacts that are related to the stream owner.
If developer Adam posts a comment on a task owned by developer Eve, then Eve's stream would contain an event showing that Adam commented on her task.
Similarly, the stream of a developer also contains information about relevant code modifications that overlap or potentially interact with code he or she previously modified.


Overall all these recommender systems provide awareness of who might be worth to interact with.
None of those systems are aiming at a concrete goal to accomplish other than achieving awareness.
We think that a focus is needed, such as on awareness with respect to dependencies that are relevant for build success.
Without such a focus the information that a developer needs to survey can quickly take up to much precious development time and may lead a developer to abandon those systems as they are taking up more time than they save.



\section{icse - Background}
\label{ch5:bg}
Before we start exploring the research question using data we obtained from the Rational Team Concert Team, we discuss related work investigating the relationship between team communication with respect to software integration issues as well as software quality and thus further motivating the value in investigating the relationship between communication and build success.

\subsection{Motivation}
Communication problems can lead to coordination and integration failures in work
teams (e.g.~\cite{Grinter:1999geography,Herbsleb:1999ew,souza:cscw:2004}). This
situation is further exacerbated in large and distributed teams, where effective
communication and activity awareness of related but remote project members are
problematic, yet key to anticipate and resolve coordination problems early
(e.g.~\cite{Grinter:1999geography,Herbsleb:1999ew}).

We strive to contribute to the growing body of research into the role of
communication structures in determining coordination ease~\cite{hinds:cscw:2006} and
success~\cite{hossain:cscw:2006}. Although there has been some research on the
relationship between communication and coordination in work teams, research in
software engineering is very limited. The study of the Enron email corpus found
that central communicators exhibit a better ability to
coordinate~\cite{hossain:cscw:2006} and R\&D teams with dense communication
structures are associated with more coordination problems~\cite{hinds:cscw:2006}. In
open source software development, developers that are active in email
communication have also been found to be most active in open source
development~\cite{bird:msr:2006}. In software engineering research, however it is
unclear whether there are specific communication behaviors that enable effective
coordination. Moreover, we are missing a precise conceptualization and objective
measure of what successful communication in relation to project success is.

Complementary to previous research that is largely qualitative
\cite{herbsleb2003:speed,Holmstrom:2006gd} and which gathered information about
occurrences of communication problems through project reviews and subjective
ratings of project success, we use objective measures in studying the
relationship between communication structures and coordination success. Past
research has also largely investigated communication and coordination only in
relation to entire projects. There is little systematic software engineering
research in objectively examining the outcome of coordination (successful or
failed) and assessing communication characteristics that lead to coordination
failures. In this work we investigate the relationship between communication
structures and coordination outcome at a finer level of detail. We study
instances of coordination during the integration of code in large and distributed
software teams, in relation to their associated communication structures. By
communication structure we refer to the topology of the communication network
that was involved in the tasks that lead to a software build. We use social
network analysis measures such as density and centrality to obtain measurable
characteristics of the communication structure.


\subsection{Communication, Coordination and Integration}
\label{sec:RelatedCommunication}
The relationship between communication, coordination and project outcome has been
studied for a long time in the area of computer-supported cooperative work. More
recently the domain of software and distributed software development showed
increased interest as well.

Communication plays an important role in work groups with high coordination needs
and the quality of communication has been found as determinant of project
success~\cite{curtis:acm:1988,kraut:1995coordination}. The dynamic nature
of work dependencies in software development makes collaboration highly
volatile~\cite{Cataldo:2007hb}, consequently affecting a teams ability to
effectively communicate and coordinate. Additional difficulties emerge in
distributed teams, where team membership and work dependencies become even more
invisible~\cite{damian:icgse:2007}. Moreover, team communication patterns are
significantly affected by distance~\cite{hinds:cscw:2006}. Maintaining
awareness~\cite{sarma:2006icgse} becomes even more difficult when developers work
in geographically remote environments; communication structures that include key
contact people at each site are effective coordination strategies when
maintaining personal cross-site relationships is challenging~\cite{hinds:cscw:2006}.

With respect to the role of effective coordination in project success, early
studies indicate the issues that software development teams face in large
projects~\cite{curtis:acm:1988}. A study by Herbsleb et
al.~\cite{Herbsleb:1999ew} showed that Conway's law is also applicable for the
coordination within development teams, supporting the influence of coordination
on software projects. Kraut et al.~\cite{kraut:1995coordination} showed that
software projects are greatly influenced by the quality of coordination of
development teams. More recently a theory of coordination has been proposed and
accounts for the influence of coordination on different project metrics such as
rework and defects~\cite{Herbsleb:2006vn}.


The importance of communication in successful coordination is also well
documented and makes the study of communication structures important. For
example, Fussell et al.~\cite{fussell:cscw:1998} found that communication amount and
tactics were linked to the ability of effectively coordinate in work groups. In
software development, others showed that communication problems lead to problems
during the activity of subsystem
integration~\cite{Grinter:1999geography,deSouza2004:thwarts_collaboration}. Coordination
conceptualized via communication has also been studied more generally in relation
to project success: factors such as ``harmony''~\cite{Souder:1988jpim},
communication structure~\cite{Robin:1990jpim}, and communication
frequency~\cite{Griffin:1992ms} were related to project success.

The difficulty in studying failed integration in relation to communication lies
in capturing and quantifying information about communication in teams that have a
well-defined coordination goal but dynamic patterns of interaction. In our work
we use the Jazz project data, which captures communication of project
participants. This enables us to study the structure of the communication
networks emerged around code integrations, both at individual teams of the
project and within the entire project.

\subsection{Can communication predict failure?}
\label{sec:ResearchQuestions}
In order to examine the communication involved in the coordination necessary
during subsystem integrations, we draw on social network analysis methods. Social
network analysis has often been deployed to study communication networks of work
teams. Using social network analysis has the major advantage that we can draw
from its extensive knowledge of analysis and implications with respect to social,
communication, and knowledge management
processes~\cite{Burt:1995vo,Freeman:1979rl}. Griffin and
Hauser~\cite{Griffin:1992ms} investigated social networks in manufacturing teams.
They found that a higher connectivity between engineering and marketing increases
the likelihood of a successful product. Similarly, Reagans and
Zuckerman~\cite{RayReagans:2001os} related higher perceived outcomes to denser
communication networks in a study of research and development teams.

Communication structure in particular -- the topology of a communication network
-- has been studied in relation to coordination
(e.g.~\cite{hossain:cscw:2006,hinds:cscw:2006}) and a number of common measures of
communication structure include network density, centrality and structural
holes~\cite{Wasserman:1994sq,Freeman:1979rl}.

Density, as a measure of the extent to which all members in a team are
connected to one another, reflects the ability to distribute
knowledge~\cite{Rulke:2000ys}. Density has been studied, for example, in relation
to coordination ease~\cite{hinds:cscw:2006}, coordination
capability~\cite{hossain:cscw:2006} and enhanced group
identification~\cite{RayReagans:2001os}.


Centrality measures indicate importance or prominence of actors in a
social network. The most commonly used centrality measures include degree and
betweenness centrality having different social implication. Centrality measures
have been used to characterize and compare different communication networks
constructed from email correspondence of W3C (WWW consortium) collaborating
working groups developing new technical standards and architectures for the
web~\cite{Gloor:2003cikm}. Similarly, Hossain et al.~\cite{hossain:cscw:2006}
explored the correlation between centrality in email-based communication networks
and coordination, and found betweenness to be the best measure for coordination.
\emph{Betweenness} is a measure of the extent to which a team member is
positioned on the shortest path in between other two members. People in between
are considered to be ``actors in the middle'' and to have more ``interpersonal
influence'' in the
network(e.g.~\cite{Gloor:2003cikm,zimmermann:icse:2008,hossain:cscw:2006}).

The structural holes measures are concerned with the degree to which there
are missing links in between nodes and with the notion of redundancy in
networks~\cite{Burt:1995vo}. At the node level, structural holes are gaps between
nodes in a social network. At the network level, people on either side of the
hole have access to different flows of information~\cite{Hargadon:1997asq},
indicating that there is a diversity of information flow in the network.
Structural holes have been used to measure social capital in relation to the
performance of academic collaborators (e.g.~\cite{Brambila:PICMET2007}).

To further our investigation into the role played by communication in
predicting integration failure, we go one step further and investigate whether
the different communication structure measures can be combined into a prediction
model that indicates whether an integration will fail.

Past research on failure prediction was not able to find a single code or code
churn metric predicting
failures~\cite{nagappan:icse:2006,basili:1996tse,denaro:2002seke}, though the
combination of those measurements became a strong predictor
(e.g.~\cite{mockus:2000bell}). This leads us to believe that even if we do not
find a single communication structure measure that predicts integration outcome,
it is useful to combine the communication network measures -- as a reflection of
the communication structure of a team -- into a predictive model and study its
predictive power.

Most prediction models in software engineering to date mainly leverage source
code related data and focus on predicting failing software components or failure
inducing changes
(e.g.~\cite{bell:2005tse,schroeter:isese:2006,zimmermann:icse:2008,kim:2008tse}).
Only few studies, such as Hassan and Zhang~\cite{hassan:ase:2006}, stepped away
from predicting component failures and used statistical classifiers to predict
integration outcome. Recently, we can observe a trend towards leveraging
developer networks, created upon code related dependencies, to predict component
failures~\cite{pinzger:fse:2008,meneely:fse:2008}. In our work, we focus on the
team coordination as given by their communication instead of source code, and
similar to Hassan and Zhang predict integration outcome.

\section{tse - Background}
\label{sec:background}
Before we start with exploring the research question using data we obtained from the Ration Team Concert Team, we discuss related work investigating socio-technical congruence as well how the subject of team coordination has been approach by other researchers.

\subsection{Motivation}
Coordinating the efforts of individuals working together in a team is
necessary to build software systems. The complexity of current systems require contributions
from tens or hundreds of people who may span multiple offices, cities, or even continents.
To build such systems, we need to ensure that the team is not only capable of
developing components of a system, but also has the governance to be able to integrate the
interdependent parts into a whole.

We describe a case study of socio-technical coordination and its effect on software builds in the IBM
Rational Team Concert (RTC) software product.
Our approach to investigating coordination is to examine the
alignment between the technical dimension of work and the social relationships
between team members. This alignment is called socio-technical
congruence~\cite{cataldo:cscw:2006}. High socio-technical congruence
has been shown to be a predictor of coordination success
\cite{cataldo:cscw:2006,ehrlich:stc:2008}.
The mismatches between the social and technical dimensions, or gaps,
also have been observed as increasing resolution times for software activities.
The objective of this study is to investigate the effects of socio-technical congruence on high-coordination software development activities during a project.

We seek to discover the relationship that congruence has on the probability that a regularly-scheduled software build will be successful.marczak:re:2008
We conduct a case study of a large software project at IBM.  A build result, which can be \emph{error} or \emph{OK}, indicates the relative health of the project up to that build. To
measure socio-technical congruence we apply two different measures: a
previously-published congruence approach \cite{cataldo:cscw:2006},
and a weighted congruence approach that provides details about the size of a
gap between two individuals~\cite{kwan2009:weighted}. We
also examine RTC's processes and tools to identify any
explanations of the relationship between congruence and builds that
we find.


\subsection{The Need for Coordination}
Software is extremely complex because of the sheer number of dependencies~\cite{sawyer2004:teams}.
Large software projects have a large number of components that interoperate with one another.
The difficulty arises when changes must be made to the software, because a change in one component of the software often requires changes in dependent components~\cite{desouza:2008}. Because a single person's knowledge of a system is specialized as well as limited, that person often is unable to make the appropriate modifications in dependent components when a component is changed.

Coordination is defined as ``integrating or linking together different parts of an organization to accomplish a collective set of tasks''~\cite{vandeven1976}. In order to manage changes and maintain quality, developers must coordinate, and in software development, coordination is largely achieved by communicating with people who depend on the work that you do \cite{kraut:1995coordination}.

A successful software build can be viewed as the outcome of good coordination because the build requires the correct compilation of multiple, dependent files of source code.
A failed build, on the other hand, demotivates software developers \cite{holck2004,damian:icgse:2007} and destabilizes the product \cite{cusumano1997}.
While a failed build is not necessarily a disaster, it slows down work significantly while developers scramble to repair the issues.
A build result thus serves as an indicator of the health of the software project up until that point in time.

Thus, a developer should coordinate closely with individuals whose technical dependencies affect his work in order to effectively build software. This brings forth the idea of aligning the technical structure and the social interactions \cite{herbsleb2007:fose}, leading us to the foundation of socio-technical congruence.

\subsection{Coordination in Software Teams}
Research in software-engineering coordination has examined interactions among
software developers \cite{carter2004,marczak:re:2008}, how they acquire
knowledge \cite{ehrlich:icgse:2006,nakakoji2010:rdc}, and
how they cope with issues including geographical
separation~\cite{espinosa2007:team_knowledge,herbsleb2003:speed}.
The ability to coordinate has
been shown as an influential factor in customer satisfaction \cite{kraut:1995coordination} and  improves the capability to produce quality work~\cite{faraj2000}.


Software developers spend much of their time
communicating~\cite{perry94}. Because developers face
problems when integrating different components from heterogeneous environments~\cite{redmiles2007:continuous},
developers engage in direct or indirect
communication, either to coordinate their activities, or to acquire knowledge of
a particular aspect of the software ~\cite{nakakoji2010:rdc}.
Herbsleb, et al. examined the influence of coordination on integrating software
modules through interviews~\cite{herbsleb1999:architectures}, and found that
processes, as well as the willingness to communicate directly, helped teams
integrate software. De Souza, et al.~\cite{desouza2007:awarenessnetwork} found that implicit
communication is important to avoid collaboration breakdowns and delays. Ko, et al.~\cite{ko:icse:2007} found that developers were identified as the main source of knowledge about code issues.
Wolf, et al.~~\cite{wolf:icse:2009} used properties of social networks to predict the outcome of integrating the software parts within teams.
This prior work establishes the fact that developers communicate heavily about technical matters.

Coordinating software teams becomes more difficult as the distance between people increases \cite{herbsleb:icse:2001}.
Studies of Microsoft~\cite{bird2009:dds_quality,nagappan:icse:2008}
show that distance between people that work together on a
program determine the program's failure proneness.
Differences in time zones can affect the number of defects in software projects \cite{cataldo2009:quality}.

Although distance has been identified as a challenge, advances in collaborative
development environments are enabling people to overcome challenges of distance.
One study of early RTC development
shows that the task completion time is not as strongly affected by distance as in previous studies~\cite{Nguyen:2008Distance}. Technology that empowers distributed collaboration include topic recommendations~\cite{carter2004} and instant messaging~\cite{niinimaki2008}. Processes are adapting to the fast pace of software development: the Eclipse way~\cite{frost:ieeesoftware:2007} emphasizes placing milestones at fixed intervals and community involvement.


\subsection{Socio-technical Congruence}
Socio-technical congruence is defined as the match between the coordination needs established by the technical domain and the actual coordination activities carried out by project members. Socio-technical congruence in software engineering was brought to attention by Cataldo, et al.~\cite{cataldo:cscw:2006}, though the concept has been explored in engineering \cite{browning2001} and management science \cite{henderson1990}. A coordination need indicates that two persons should be coordinating based on the technical dependencies on the project. A coordination need is determined by analysing the assignments of people to a technical entity such as a source code module, and the technical dependencies among the technical entities.
Socio-technical congruence states that if there is a coordination need between two people, these people should be coordinating.
For example, if two people work on different, but dependent components of the project, then those persons should be coordinating with each other.

If two individuals have a coordination need, but do not coordinate, then there is a gap between these two individuals. A gap suggests the existence of a coordination problem. One of the goals of socio-technical congruence is to minimize the number of gaps, either by maintaining good coordination between individuals who have a coordination need, or by reducing the number of technical dependencies in the project and therefore reducing the coordination needs~\cite{sarma2008:measuring_stc}.

What socio-technical congruence offers is an approach to measure the coordination quality~\cite{cataldo:cscw:2006}. We can use this measurement to identify the effect of socio-technical coordination on software build quality.

\subsection{Effects of Socio-technical Congruence}
Current research suggests that attaining a high level of socio-technical congruence is beneficial to an organization.
Evidence shows that higher congruence leads to faster completion of modification requests~\cite{cataldo:cscw:2006}. 
The presence of gaps increases the number of code changes \cite{ehrlich:stc:2008}, and a lack of coordination connections across system and organizational boundaries have a negative effect on performance~\cite{sosa2004:manage}.

Socio-technical gaps have been found to be an issue not only because they lower
the congruence and thus lower productivity~\cite{cataldo:cscw:2006}, but because they are especially problematic in the context of distributed development~\cite{ehrlich:stc:2008}. Thus, researchers have proposed remedial actions when socio-technical congruence gaps are discovered~\cite{valetto2007:value}. 
Examples of actions include closing a gap by augmenting coordination and eliminating the gap by refactoring software.

The usefulness of socio-technical congruence depends on the conceptualizations of
the social and the technical dimensions. Communication is believed to help people coordinate. However, it is not the only way to describe the social dimension. 
For instance, Cataldo et al.~\cite{cataldo:cscw:2006} evaluated congruence in the context of software development using different representations of actual coordination, including geographical proximity, IRC communication, and issue-tracking comments; these factors correlate with the resolution time of modification requests. There are also variations in the way technical dependencies can be handled. Cataldo et al.~\cite{cataldo:esem:2008} used differing ways to measure architectural dependencies and found that the congruence values computed using a ``files changed together'' dependency are more reliable than call graph dependencies~\cite{deSouza2004:thwarts_collaboration}. Gokpinar, et al \cite{gokpinar2010} applied a congruence technique and discovered that a higher coordination deficit leads to a larger number of filed incident reports, implying reduced quality.

Socio-technical congruence has been explored outside of the software development field, particularly in engineering and management disciplines \cite{henderson1990,sosa2004:manage,gokpinar2010,sosa2008}. Sosa \cite{sosa2008} described a formal technique to compute socio-technical congruence and identified ``potentially unattended technical interactions'', which are technical dependencies in modules that are not monitored. Gokpinar \cite{gokpinar2010}, independently of our work, developed a weighted socio-technical congruence technique that he applies to the automotive industry.

\section{fse - Background}
\label{ch8:bg}
In this section we discuss some background related to improving coordination among software developers.
We start with some motivation of the problem and continue with presenting work performed by other researchers.

\subsection{Motivation}
We hypothesize that with the ever growing size of software teams the lack of
effective coordination is the main source of integration failures.  With the
ever growing complexity and sophistication of large software projects,
error-free integrations are not only important but difficult to achieve. The
development work that precedes integrations involves significant coordination of
developers that work in teams and need to rely on the code of others and its
stability. But often code is everything but stable, further contributing to
developers' need to coordinate to keep up with code changes that impact their work. This problem is
amplified in software builds where an entire team needs to integrate their work
and on which the development of new features depends. Not only do
failed builds destabilize the product~\cite{cusumano1997} but they also demotivate
software developers~\cite{holck2004}.

Despite their importance, keeping integrations builds error-free can be a very time consuming
process. A lightweight approach that can determine whether the build contains
failures before invoking the build process is thus very valuable to developers.
This lightweight approach could determine a builds outcome in minutes rather than hours or days. Having a faster way to
assess the quality of a build helps developers to continue working with newest
builds while being aware of its quality. Previous
research~\cite{wolf:icse:2009,hassan:ase:2006} trained predictive models to assess the quality of software builds without the need of invoking large test
suits. Although this research
reaches a high degree of accuracy in their predictions, knowing that a
build will fail does not necessarily help developers to actually prevent
the build from failing.
The goal of this research is to find a way to create actionable knowledge that
developers can act upon to avoid
integration failure.

Maintaining proper communication and awareness of work others perform is
important in any kind of project. Specifically in software engineering many studies found
that factors such as geographical and organizational distance have an impact on
communication and even effect software quality~\cite{nagappan:icse:2008}. In our
study we uncover the existence of pairs of
developers, that, if technically dependent in a build but not discussing their
dependencies, have a negative influence on the success on builds. This
actionable knowledge can be integrated in real-time recommender systems that
indicate, based on project historical data, which developer pairs tend to be
failure related. Developers and management can then devise strategies to
prevent the failure before build time. 


\subsection{Related Work}
\label{sec:relwork}
In order to manage changes and maintain quality, developers must coordinate. In
software development, coordination is largely achieved through communicating with
people who depend on the work that you do \cite{kraut:1995coordination}. The
software engineering literature is recognizing the role of communication as
something that should be nurtured not eliminated and recent
collaborative software development environments aim to support developers'
social interactions along with artifact creation activities~\cite{nakakoji2010:rdc}.

Ehrlich et al.~\cite{ehrlich:icgse:2006} investgiated how social networks can be
used to leverage knowledge in distributed teams. Backstrom et
al.~\cite{backstrom:kdd:2006} took a more general approach and investigated the
evolution of large social networks and the information they hold. Chung et
al.~\cite{chung:cpr:07} reported in recent work about behavior of individuals
while performing knowledge intensive tasks. There have been a number of studies
that investigated communication structures to identify good
coordination practices
(e.g.~\cite{hinds:cscw:2006,hossain:cscw:2006,bird:fse:2008,hinds:hicss:2008}). In contrast to studies of the general development process, Marczak studied social
networks to identify best practices for requirements management
processes~\cite{marczak:re:2008}.

Inspired by Conways Law~\cite{conway:datamination:1968}, Cataldo et
al.~\cite{cataldo:cscw:2006,cataldo:esem:2008} formulated a coefficient that
measures the alignment of the social and technical networks defining the term of
socio-technical congruence. They observed that higher socio-technical congruence
leads to higher developer
productivity~\cite{cataldo:cscw:2006,cataldo:esem:2008}. Others used this
notion and coefficient to further investigate the effect of congruence
(e.g.~\cite{valetto:msr:2007}). Prior to Cataldo et
al.~\cite{cataldo:cscw:2006,cataldo:esem:2008} proposal,
Ducheneaut~\cite{ducheneaut:cscw:2005} investigated the evolution of social and
technical relationships of open source project participants to see how those
participants become a part of the community.

Recent studies started to relate the social with the technical
dimensions of software development to build predictive models. Pinzger et
al.~\cite{pinzger:fse:2008} successfully used social networks connecting
developers via code artifacts to predict failures. Meneely et
al.~\cite{meneely:fse:2008} used similar networks but excluded the code artifacts
and connected the developers directly. Two studies at Microsoft looked into the
geographical~\cite{bird:acm:2009} and organizational~\cite{nagappan:icse:2008}
distance between people that worked on the same binary and the relation to the
failure proneness of said binary. They found that the organizational distance is
a very powerful predictor of failure proneness of binaries whereas the
investigation of geographical distance has little to no effect. A recent
study~\cite{bird:issre:2009} combines the work of Pinzger et
al.~\cite{pinzger:fse:2008} and
Zimmermann~\cite{zimmermann:icse:2008} by creating
socio-technical networks that capture developer contributions and binary
interdependencies. They found this combination to be a more powerful predictor
that works for different software project and even prevails across multiple
revisions of a project.

\subsection{cscw - Background}
\label{ch9:bg}
%old introduction
For all but the most trivial programs, software development today is a collaborative activity. The members of software teams must coordinate and communicate, often intensely, if their projects are to reach a satisfactory conclusion.
Previous research has shown that communication among developers has a profound influence on several aspects of software development~\cite{hinds:cscw:2006,wolf:icse:2009}.

Despite this acknowledged importance of communication in software teams, our understanding of communication behaviour is still very limited. Ko \emph{et al.}~\cite{ko:icse:2007} point out that developers often require information about the work output of their peers.  We know little about why developers act upon this requirement. What causes developers to seek information in some cases, but not in others? There are a multitude of potential factors: the closeness of their peers' work with their own, the expertise of the developers, their physical proximity, their level of workload, and even, perhaps, the time of day. By identifying which factors actually influence the likelihood that a developer will request information about the events surrounding her, we might better understand the nature of software-centric information-seeking behaviour. This increased understanding allows us to develop improved recommender systems that enable developers to coordinate their activities efficiently.

% related work start
Overall, this research builds upon the growing body of work that studies coordination and communication in software organizations, which have long been recognized as playing central roles in leading software projects to success, e.g.~\cite{kraut:1995coordination,curtis:acm:1988}.  

Earlier we reported on investigations of the RTC project repositories revealed a significant relationship between gaps in communication structures and build success (Chapter~\ref{chap:stc-net2} and~\ref{chap:stc-net}). Therefore we proposed a recommendation system that allows team members to learn about who they should talk with in order to avoid build failures~\cite{schroeter:rsse:2008}. This recommendation system should focus on changesets and analyze significant differences between code dependencies and team communication structure to identify anti-patterns that the developers should attempt to break.
The basis for this principle is derived from Conway's Law~\cite{conway:datamination:1968} and by the model of socio-technical congruence proposed by Cataldo et al~\cite{cataldo:cscw:2006,cataldo:esem:2008}.
The model compares the communication network to the code dependency network and derives insights with respect to productivity from the gaps between those networks.

In order to build a good recommendation system, however, one needs to go beyond providing valuable information. As Murphy and Murphy-Hill show~\cite{murphy:rsse:2010}, we need developers to trust the recommender system: delivering the right recommendation at the right time is one important way to build such trust. Unfortunately, recommender systems often fall short on this front, saturating the developer with recommendations they neither need nor want, or presenting recommendations at inappropriate times. It is therefore important to know when developers actually need information, and what factors influence them to seek information. There are many potential factors that could change the likelihood that developers would want to obtain further information, and we do not have a sense, other than our intuition, to learn which factors are actually important and which are not.

Related work has examined factors that influence communication among software developers. For instance, while APIs are intended to reduce coordination complexity among development teams, they also introduce barriers that hinder communication between teams by obscuring dependencies~\cite{souza:cscw:2004,desouza:fse:2004}.
Software architecture, organizational structure and project age are also factors that influence communication and the awareness that developers have of each other's work~\cite{cleidson:tse:2011}. 
Processes, in particular, are often used to influence the communication among team members to reduce information overload~\cite{fussell:cscw:1998}, and geographic distance has been shown to affect communication, but its effects vary between projects~\cite{herbsleb:icse:2001,wolf:spip:2008}.
Nakakoji et al~\cite{nakakoji2010:rdc} distilled the current knowledge about communication among developers into a set of guidelines to design tools and processes that facilitate effective expertise communication.
Those studies, and to our knowledge, literature in software engineering in general, only address communication on a project-wide level. However, to fully understand, and thus offer actionable knowledge to developers, we need to know what influences an individual developer's information-seeking behaviour.
We surveyed literature to compile a list of possible information-seeking triggers spanning change-related (such as code metrics), developer-related (such as experience level) and process-related (such as peer reviews) factors.










\section{Next Steps/Research Questions}
The concept of socio-technical congruence shows potential to help make software development more efficient.
Cataldo et al~\cite{cataldo:cscw:2006} demonstrated its relation to productivity, and we show among other things in this thesis the ability to use socio-technical congruence to predict software quality.
The concept of socio-technical congruence lends itself to improve software development as it is based on social networks connecting developer on a coordination and technical level.
Because of the concept being based on networks it is possible to manipulate the networks.

There are two steps that need to be taken:
(1) identify the parts of the network that need to be changed
and (2) devise strategies to change the social network in a timely fashion.
In case of failure prevention the current way socio-technical congruence is constructed might be sufficient in that it is enough to compute after a task has been completed to assess the quality of the submitted code.
On the other hand, to contribute to the improvement of productivity we cannot afford to use networks constructed from submitted work as the work that we want to speed up has already been submitted.
Even though in the case of software quality it is sufficient to assess the work done to accomplish a task, providing earlier feedback is advantageous as it allows developer to intervene early on and spend less time on correct errors later on.

Any socio-technical network can be manipulated in two ways: (1) changing the technical dependencies among developer by refactoring or architectural changes to make them unnecessary and (2) by engaging developer in discussions about their recent work and therefore creating a coordination edge in the socio-technical network.
Since many product are not developed from scratch and because architectural changes once development has been going on for a number of months are costly and time consuming~\cite{vangurp:jss:2002}, we aim at generating recommendations to change the actual coordination to improve the socio-technical network where it matters.
Therefore, as a first step we need to assess if the actual communication structure among software developers has an influence on build success to lay the basis for manipulating the actual coordination to increase build success.
As a follow up step, we need to explore the relationship between socio-technical networks and build success.
Especially we are interested in the relationship in whether missing actual coordination in the face of a coordination needs is related to build success.

We start in the second part of this thesis with investigating the influence of communication among team members in the form of the communication structure represented by social networks on build success.
Next, we investigate if gaps (unfilled coordination needs) between developers as highlighted by socio-technical networks and the socio-technical networks themselves can be brought into relation with build success.
Therefore Chapter~\ref{chap:soc-net} and~\ref{chap:stc-net2} investigate the following two research questions respectively:

\begin{description}
  \item[RQ 1.1:] Do Social Networks influence build success? (Chapter~\ref{chap:soc-net})
  \item[RQ 1.2:] Does Socio-Technical Networks influence build success? (Chapter~\ref{chap:stc-net2})
\end{description}

Having found a relationship between socio-technical networks, especially gaps between coordination and coordination needs, with build success, while knowing that communication alone has an effect on build success we formulate an approach to leverage socio-technical networks in Chapter~\ref{chap:approach}.
The third and final part of this thesis focuses on evaluating this approach in three ways:
(1) gathering general statistical evidence of parts of the network and build success to foster recommendations,
(2) exploring the acceptance of such recommendation by developers,
and (3) a proof of concepts that the recommendation could prevent failures.
Hence, the first three chapters of the third part of this thesis are guided by the following three research questions:

\begin{description}
  \item[RQ 2.1:] Can Socio-Technical-Networks be manipulated to increase build success? (Chapter~\ref{chap:stc-net})
  \item[RQ 2.2:] Do developer accept recommendations based on software changes to increase build success? (Chapter~\ref{chap:talk})
  \item[RQ 2.3:] Can recommendation actually prevent build failures? (Chapter~\ref{chap:actionable})
\end{description}

In the following discussion Chapter~\ref{chap:disc} we will highlight how our findings from these three research questions support the approach we detailed in Chapter~\ref{chap:approach}.









