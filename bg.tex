\startchapter{Background}
In this chapter we will give a rough overview over the challenge that we described in the introduction and how other work tried partially or in full to address these challenges.
Note that we will not go into great detail of the indvidual steps that we undertook in this thesis as we discuss related work in the appropriate chapter.
Thus we will discuss two areas: (1) the research around the concept of socio-technical congruence and (2) recommender systems in sofftware engineering.

\section{Socio-Technical Congruence}
As mentioned earlier this thesis explores to what extend cane we leverage the concept of socio-technical congruence, before we start discussion the work that has been done with respect to using the concept of socio-technical congruence to analyise softare development teams and their performance, we explain the socio-technical congruence conpect and it different ways it was implemented so far.
We close this section by discussion the next step we are going to take with respect to concept.

\subsection{Socio-Technical Congruence Definitions}
\subsection{Socio-Technical Congruence and Performance}
\subsection{Socio-Technical Congruence Next Steps}

\section{Recommender Systems in Software Engineering}
