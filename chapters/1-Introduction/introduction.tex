\startfirstchapter{Introduction}
The software industry often visible through some of their biggest companies such as Microsoft, Google, IBM, Dell, Apple, Oracle, and SAP represents several hundred billion US Dollars a year. 
For example \todo{insert some statistics about countries se industry sizes} U.S.A, Canada, Germany.
As many engineering companies those companies in the software industry strive to optimize their engineering processes to produce software of higher quality in less time.

Software engineering researchers all over the world have dedicated countless hours dedicated to improve the way software is developed.
Several fields reaching not directly aimed at increasing productivity such as developing better programming languages~\cite{}, smarter compilers~\cite{}, and better education in algorithms and data structures~\cite{}.
Other fields are more directly interested in productivity, among them are research in software processes~\cite{}, effort estimation~\cite{}, and software failure prediction~\cite{}.

The fast body of knowledge accumulated to improve the software engineering process is strongly biased towards analyzing the technical side: supporting coding activities~\cite{} and analyzing source code to improve quality~\cite{}. 
Since producing source code is the main objective of software developer, optimizing the coding aspect~\cite{} as well as analyzing the produced code for issues~\cite{} lies at hand.

Others have focused on the people that produce the code. Studying their behavior around coding activities~\cite{}, how they communicate~\cite{} and how their relations relate to productivity~\cite{}.
As in the former case there is much merit in focusing on the developer in the end she implements the features a software consists of and she inevitably introduces errors to the code base.

Both avenues, studying the human aspect and studying the technical aspect, yielded many useful results.
For example,  
