\startchapter{Motivation and Research Questions}
Software engineering as other engineering disciplines is concerned with making the process of developing (or engineering) software more efficient.
As software development is a human intensive task, reducing the time spend on any given piece of software, either creating or maintaining it, directly reduces costs.
Both creation and maintenance are both substantially influenced by quality issues discovered by testing the software or reported by customers.
In either case, creation or maintenance, getting the software into a state it can be shipped to the customer is the next important goal of any team.
In this thesis we are focusing on improving upon quality with a focus on improving the success of creating a sufficiently working version of product commonly referred to as build.

The concept of Socio-Technical congruence introduced by Conway~\cite{} especially the refined version by Cataldo et al~\cite{} offers a good starting point to explore why software builds might fail.
Socio-Technical congruence is composed of both the technical relations within a piece of software and the corresponding social relations between developer within a project.
This concept both builds upon existing work of technical as well as social factors that relate to software quality.

In the first half of this thesis we will discuss the impact on social, technical and socio-technical networks on build success.
To understand if it makes sense to look at the concept of socio-technical congruence, which was first conceived to relate to overall productivity instead of software quality, we first need to established if the separate parts of socio-technical congruence namely technical and social factors influence build success, our software quality metric of choice.

Since the end goal is to establish a measure that can be made actionable it is important that at least one of the two sides of the socio-technical concept, social or technical, has an effect on software builds.
Unless this effect exists it will be difficult to manipulate the socio-technical networks to improve the success of software builds.
Therefore we formulate the research questions guiding the first half of this thesis (Chapters~\ref{}-\ref{}) as follows:

\begin{description}
% 
\item[RQ 1:] Does Socio-Technical Congruence influence build success?
  %
  \begin{description}
  \item[RQ 1.1:] Do Technical Networks influence build success?
  \item[RQ 1.2:] Do Social Networks from repositories influence build success?
  \item[RQ 1.3:] Does Socio-Technical Networks influence build success?
  \end{description}
%
\end{description}
Research Question 1.1 explores through a literature review in Chapter~\ref{} how technical relationships or the relationships between the units of work produced by a developer influence software quality.



In the second half of this thesis we will discuss

\begin{description}
%
\item[RQ 2:] Can Socio-Technical Networks be used to recommend improvements? 
  %
  \begin{description}
  \item[RQ 2.1:] Can Socio-Technical-Networks be used to suggest improvements that matter?
  \item[RQ 2.2:] Does recommending on the lowest level make sense?
  \item[RQ 2.3:] When does recommending on the lowest level make sense?
  \item[RQ 2.4:] Can recommendation be given in a timely manner?
  \end{description}
\end{description}