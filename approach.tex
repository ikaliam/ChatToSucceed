% !TEX root = thesis.tex
\startchapter{The Approach}
\label{chap:approach}
In this chapter we will propose an approach to leverage the concept of socio-technical congruence to improve communication (social) interactions among developer to improve build success.
We base this approach on the findings presented in the previous two chapter that team communication (Chapter~\ref{}) and socio-technical gaps (Chapter~\ref{}) have influence on build success.

\section{Approach to Leveraging Socio-Technical Congruence}
The main contribution of this thesis lies with the implicitly described approach to creating actionable knowledge from the socio-technical congruence concept as Cataldo et al~\cite{cataldo:cscw:2006} described it in their seminal paper.
Since the research questions we formulated for this thesis only implicitly describe the approach, it is necessary to describe the approach to wrest actionable insights from the concept of socio-technical congruence.

Our approach took five steps:
\begin{enumerate}
\item Define scope of interest.
\item Define outcome metric.
\item Build social networks according to the scope in real time.
\item Build technical networks according to the scope in real time.
\item Generate actionable insights.
\end{enumerate}

\paragraph{Scope} 
Each network, social, technical and thus socio-technical, needs to build with a specific scope in mind.
For example, throughout this thesis we used software builds a focus.
This focus defines the source and communication artifacts that are used to construct the socio-technical networks. 

\paragraph{Outcome}
In order to derive useful insights from the constructed networks the scope used to construct them needs to be complimented with an outcome metric.
For example, throughout this thesis we used build success, a binary variable that states whether a build is of acceptable quality.
With an outcome measure at hand we can contrast networks to gain insights.

\paragraph{Social Networks}
After the defining the scope and outcome, the next step is to construct the socio-technical networks.
In our approach we first focus on the construction of the social part of the networks.
This involves identifying all communication channels that are programatically accessible in real time.
Some examples of communication channels that can be tapped into in real time are emails, forum style discussions, or text chats.
Gathering all to the selected scope relevant communication artifacts than yields the social-network.

\paragraph{Technical Networks}
To complement the social networks and thus create socio-technical networks we need to produce technical networks.
The main issues is not to rely on information that is only available after the work has been completed and been submitted to a version repository, but to gather information to construct networks in real time.
For instance, Blincoe et al~\cite{blincoe:cscw:2012} proposed to use Mylyn\footnote{\url{http://tasktop.com/mylyn/}} events to accomplish the extraction of interactions with source code in real time.

\paragraph{Insights}
In Chapter~\ref{chap:stc-net} and~\ref{chap:actionable} we showed how to generate recommendations by contrasting the networks according to their outcome metrics based on a binary outcome metric indicating whether a build failed.
By breaking down networks, as demonstrated in Chapter~\ref{chap:stc-net} and~\ref{chap:actionable}, and correlating the different elements with the outcome metric generates insights that can be used to improve the collaboration represented by the network.

In the next section we add to the example given in this section that are based on the chapters in this thesis arguments to the feasibility and validity of the individual steps to ground the approach in the research we conducted in this thesis.