% !TEX root = ../thesis.tex
\newpage
\TOCadd{Abstract}

\noindent \textbf{Supervisory Committee}
\tpbreak
\panel

\begin{center}
\textbf{ABSTRACT}
\end{center}
Efficient coordination among software developers is one key aspect in producing high quality software on time and on budget.
Many factors such as team distribution or the structure of the organization developing the software can increase the level of difficulty in coordination.
However, the problem runs deeper, it is often unclear which developers should coordinate their work.

In this thesis we propose to leverage the concept of socio-technical congruence (which contrasts coordination needs with actual coordination) to improve the social interactions among developers 
by developing an approach and its implementation into a recommender system that identifies relevant coordinatiors.
Our unit of analysis is the integration build whose outcome represents the quality of coordination.
We developed and applied and Approach in a number of case studies of the IBM Rational Team Concert development team as well as a large student project at the University of Victoria, Canada, and Aalto University, Finland.

Each software product is just the last integration build before the product was released thus ensuring a failure free build is of utmost importance to industry.
While developing an approach to improve coordination among software developers, we uncovered that unmet coordination needs as well as the communication structure in a team significantly influence build outcome.